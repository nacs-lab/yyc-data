\documentclass[10pt,fleqn]{article}
% \usepackage[journal=rsc]{chemstyle}
% \usepackage{mhchem}
\usepackage{amsmath}
\usepackage{amssymb}
\usepackage{amsfonts}
\usepackage{esint}
\usepackage{bbm}
\usepackage{amscd}
\usepackage{picinpar}
\usepackage{graphicx}
\usepackage{tikz}
\usepackage{tikz-3dplot}
\usepackage{indentfirst}
\usepackage{wrapfig}
\usepackage{units}
\usepackage{textcomp}
\usepackage[utf8x]{inputenc}
% \usepackage{feyn}
\usepackage{feynmp}
\usepackage{xkeyval}
\usepackage{xargs}
\usepackage{verbatim}
\usepackage{pgfplots}
\usetikzlibrary{
  arrows,
  calc,
  decorations.pathmorphing,
  decorations.pathreplacing,
  decorations.markings,
  fadings,
  positioning,
  shapes
}

\DeclareGraphicsRule{*}{mps}{*}{}
\newcommand{\ud}{\mathrm{d}}
\newcommand{\ue}{\mathrm{e}}
\newcommand{\ui}{\mathrm{i}}
\newcommand{\res}{\mathrm{Res}}
\newcommand{\Tr}{\mathrm{Tr}}
\newcommand{\dsum}{\displaystyle\sum}
\newcommand{\dprod}{\displaystyle\prod}
\newcommand{\dlim}{\displaystyle\lim}
\newcommand{\dint}{\displaystyle\int}
\newcommand{\fsno}[1]{{\!\not\!{#1}}}
\newcommand{\eqar}[1]
{
  \begin{align*}
    #1
  \end{align*}
}
\newcommand{\texp}[2]{\ensuremath{{#1}\times10^{#2}}}
\newcommand{\dexp}[2]{\ensuremath{{#1}\cdot10^{#2}}}
\newcommand{\eval}[2]{{\left.{#1}\right|_{#2}}}
\newcommand{\paren}[1]{{\left({#1}\right)}}
\newcommand{\lparen}[1]{{\left({#1}\right.}}
\newcommand{\rparen}[1]{{\left.{#1}\right)}}
\newcommand{\abs}[1]{{\left|{#1}\right|}}
\newcommand{\sqr}[1]{{\left[{#1}\right]}}
\newcommand{\crly}[1]{{\left\{{#1}\right\}}}
\newcommand{\angl}[1]{{\left\langle{#1}\right\rangle}}
\newcommand{\tpdiff}[4][{}]{{\paren{\frac{\partial^{#1} {#2}}{\partial {#3}{}^{#1}}}_{#4}}}
\newcommand{\tpsdiff}[4][{}]{{\paren{\frac{\partial^{#1}}{\partial {#3}{}^{#1}}{#2}}_{#4}}}
\newcommand{\pdiff}[3][{}]{{\frac{\partial^{#1} {#2}}{\partial {#3}{}^{#1}}}}
\newcommand{\diff}[3][{}]{{\frac{\ud^{#1} {#2}}{\ud {#3}{}^{#1}}}}
\newcommand{\psdiff}[3][{}]{{\frac{\partial^{#1}}{\partial {#3}{}^{#1}} {#2}}}
\newcommand{\sdiff}[3][{}]{{\frac{\ud^{#1}}{\ud {#3}{}^{#1}} {#2}}}
\newcommand{\tpddiff}[4][{}]{{\left(\dfrac{\partial^{#1} {#2}}{\partial {#3}{}^{#1}}\right)_{#4}}}
\newcommand{\tpsddiff}[4][{}]{{\paren{\dfrac{\partial^{#1}}{\partial {#3}{}^{#1}}{#2}}_{#4}}}
\newcommand{\pddiff}[3][{}]{{\dfrac{\partial^{#1} {#2}}{\partial {#3}{}^{#1}}}}
\newcommand{\ddiff}[3][{}]{{\dfrac{\ud^{#1} {#2}}{\ud {#3}{}^{#1}}}}
\newcommand{\psddiff}[3][{}]{{\frac{\partial^{#1}}{\partial{}^{#1} {#3}} {#2}}}
\newcommand{\sddiff}[3][{}]{{\frac{\ud^{#1}}{\ud {#3}{}^{#1}} {#2}}}
\usepackage{fancyhdr}
\usepackage{multirow}
\usepackage{fontenc}
% \usepackage{tipa}
\usepackage{ulem}
\usepackage{color}
\usepackage{cancel}
\newcommand{\hcancel}[2][black]{\setbox0=\hbox{#2}%
  \rlap{\raisebox{.45\ht0}{\textcolor{#1}{\rule{\wd0}{1pt}}}}#2}
\pagestyle{fancy}
\setlength{\headheight}{67pt}
\fancyhead{}
\fancyfoot{}
\fancyfoot[C]{\thepage}
\fancyhead[R]{}
\renewcommand{\footruleskip}{0pt}
\renewcommand{\headrulewidth}{0.4pt}
\renewcommand{\footrulewidth}{0pt}

\newcommand\pgfmathsinandcos[3]{%
  \pgfmathsetmacro#1{sin(#3)}%
  \pgfmathsetmacro#2{cos(#3)}%
}
\newcommand\LongitudePlane[3][current plane]{%
  \pgfmathsinandcos\sinEl\cosEl{#2} % elevation
  \pgfmathsinandcos\sint\cost{#3} % azimuth
  \tikzset{#1/.estyle={cm={\cost,\sint*\sinEl,0,\cosEl,(0,0)}}}
}
\newcommand\LatitudePlane[3][current plane]{%
  \pgfmathsinandcos\sinEl\cosEl{#2} % elevation
  \pgfmathsinandcos\sint\cost{#3} % latitude
  \pgfmathsetmacro\yshift{\cosEl*\sint}
  \tikzset{#1/.estyle={cm={\cost,0,0,\cost*\sinEl,(0,\yshift)}}} %
}
\newcommand\DrawLongitudeCircle[2][1]{
  \LongitudePlane{\angEl}{#2}
  \tikzset{current plane/.prefix style={scale=#1}}
  % angle of "visibility"
  \pgfmathsetmacro\angVis{atan(sin(#2)*cos(\angEl)/sin(\angEl))} %
  \draw[current plane] (\angVis:1) arc (\angVis:\angVis+180:1);
  \draw[current plane,dashed] (\angVis-180:1) arc (\angVis-180:\angVis:1);
}
\newcommand\DrawLatitudeCircleArrow[2][1]{
  \LatitudePlane{\angEl}{#2}
  \tikzset{current plane/.prefix style={scale=#1}}
  \pgfmathsetmacro\sinVis{sin(#2)/cos(#2)*sin(\angEl)/cos(\angEl)}
  % angle of "visibility"
  \pgfmathsetmacro\angVis{asin(min(1,max(\sinVis,-1)))}
  \draw[current plane,decoration={markings, mark=at position 0.6 with {\arrow{<}}},postaction={decorate},line width=.6mm] (\angVis:1) arc (\angVis:-\angVis-180:1);
  \draw[current plane,dashed,line width=.6mm] (180-\angVis:1) arc (180-\angVis:\angVis:1);
}
\newcommand\DrawLatitudeCircle[2][1]{
  \LatitudePlane{\angEl}{#2}
  \tikzset{current plane/.prefix style={scale=#1}}
  \pgfmathsetmacro\sinVis{sin(#2)/cos(#2)*sin(\angEl)/cos(\angEl)}
  % angle of "visibility"
  \pgfmathsetmacro\angVis{asin(min(1,max(\sinVis,-1)))}
  \draw[current plane] (\angVis:1) arc (\angVis:-\angVis-180:1);
  \draw[current plane,dashed] (180-\angVis:1) arc (180-\angVis:\angVis:1);
}
\newcommand\coil[1]{
  {\rh * cos(\t * pi r)}, {\apart * (2 * #1 + \t) + \rv * sin(\t * pi r)}
}
\makeatletter
\define@key{DrawFromCenter}{style}[{->}]{
  \tikzset{DrawFromCenterPlane/.style={#1}}
}
\define@key{DrawFromCenter}{r}[1]{
  \def\@R{#1}
}
\define@key{DrawFromCenter}{center}[(0, 0)]{
  \def\@Center{#1}
}
\define@key{DrawFromCenter}{theta}[0]{
  \def\@Theta{#1}
}
\define@key{DrawFromCenter}{phi}[0]{
  \def\@Phi{#1}
}
\presetkeys{DrawFromCenter}{style, r, center, theta, phi}{}
\newcommand*\DrawFromCenter[1][]{
  \setkeys{DrawFromCenter}{#1}{
    \pgfmathsinandcos\sint\cost{\@Theta}
    \pgfmathsinandcos\sinp\cosp{\@Phi}
    \pgfmathsinandcos\sinA\cosA{\angEl}
    \pgfmathsetmacro\DX{\@R*\cost*\cosp}
    \pgfmathsetmacro\DY{\@R*(\cost*\sinp*\sinA+\sint*\cosA)}
    \draw[DrawFromCenterPlane] \@Center -- ++(\DX, \DY);
  }
}
\newcommand*\DrawFromCenterText[2][]{
  \setkeys{DrawFromCenter}{#1}{
    \pgfmathsinandcos\sint\cost{\@Theta}
    \pgfmathsinandcos\sinp\cosp{\@Phi}
    \pgfmathsinandcos\sinA\cosA{\angEl}
    \pgfmathsetmacro\DX{\@R*\cost*\cosp}
    \pgfmathsetmacro\DY{\@R*(\cost*\sinp*\sinA+\sint*\cosA)}
    \draw[DrawFromCenterPlane] \@Center -- ++(\DX, \DY) node {#2};
  }
}
\makeatother
\tikzstyle{snakearrow} = [decorate, decoration={pre length=0.2cm,
  post length=0.2cm, snake, amplitude=.4mm,
  segment length=2mm},thick, ->]
%% document-wide tikz options and styles
\tikzset{%
  >=latex, % option for nice arrows
  inner sep=0pt,%
  outer sep=2pt,%
  mark coordinate/.style={inner sep=0pt,outer sep=0pt,minimum size=3pt,
    fill=black,circle}%
}
\addtolength{\hoffset}{-1.3cm}
\addtolength{\voffset}{-2cm}
\addtolength{\textwidth}{3cm}
\addtolength{\textheight}{2.5cm}
\renewcommand{\footskip}{10pt}
\setlength{\headwidth}{\textwidth}
\setlength{\headsep}{20pt}
\setlength{\marginparwidth}{0pt}
\parindent=0pt
\begin{document}

\section{Goal}
In the simulation of the single atom in a potential driven by multiple laser
field, we need to compute the effect of optical drive on the atom. This is
typically done in textbook by going to the rotating frame (co-rotating with the
laser field) in which the Hamiltonian is time independent. However, this does
not work in general if there are multiple laser drive frequencies. Additionally,
we would like to evaluate each drive individually (doing it otherwise requires
matrix exponentiation of a large time dependent matrix) so each drive needs to
be evaluated on top of the transformation due to internal and potential
energies (since these should be evaluated once instead of for each drive).
Therefore, the goal of this note is to derive the local propagator which
describe the evolution on top of trivial part due to the energy of each
state.

\section{Math}

With $\varepsilon$ energy different between the ground and excited state, the
Hamiltonian with a single drive of detuning $\delta_0$ is ($\hbar = 1$)

\eqar{
  H=&\frac\varepsilon2\sigma_z + \Omega\paren{\cos(\delta_0t+\phi_0)\sigma_x+\sin(\delta_0t+\phi_0)\sigma_y}
}

Our goal is to derive $T'$ such that the wave function at time $t$,

\eqar{
  |\psi\rangle_t=&T_0T'|\psi\rangle_0
}

where $|\psi\rangle_0$ is the initial state and $T_0\equiv\ue^{-\ui\sigma_z\varepsilon t/2}$ is the trivial part of the propagator.\\

In order to do this, we follow the textbook procedure of going into the rotating frame and transform the propagator for the time independent equation back to our original basis. We start from Schroedinger equation

\eqar{
  \ui\partial_t|\psi\rangle=&H|\psi\rangle
}

We can go to a rotating frame by doing the transformation
$|\psi\rangle=U|\psi'\rangle$, where $U=\ue^{\ui\sigma_z(\omega't+\phi')}$ with the
parameters $\omega'$ and $\phi'$ which we will determine later.

\eqar{
  \ui\partial_t\paren{U|\psi'\rangle}=&HU|\psi'\rangle\\
  \ui\partial_t|\psi'\rangle=&U^\dagger HU|\psi'\rangle-\ui U^\dagger\paren{\partial_t U}|\psi'\rangle\\
  U^\dagger\paren{\partial_t U}=&\ui\sigma_z\omega'U^\dagger U\\
  =&\ui\sigma_z\omega'\\
  \ui\partial_t|\psi'\rangle=&U^\dagger HU|\psi'\rangle
  +\sigma_z\omega'|\psi'\rangle
}

For the $U^\dagger HU$ term, we have,

\eqar{
  \sqr{U, \sigma_z}=&0
  \intertext{And,}
  &\ue^{-\ui\sigma_z\varphi}\paren{\cos\theta\sigma_x+\sin\theta\sigma_y}\ue^{\ui\sigma_z\varphi}\\
  =&\ue^{-\ui\sigma_z\varphi}\paren{\cos\theta\sigma_x+\sin\theta\sigma_y}\paren{\cos\varphi+\ui\sigma_z\sin\varphi}\\
  =&\ue^{-\ui\sigma_z\varphi}\paren{\sigma_x\cos\theta\cos\varphi-\sigma_x\sin\theta\sin\varphi+\sigma_y\cos\theta\sin\varphi+\sigma_y\sin\theta\cos\varphi}\\
  =&\ue^{-\ui\sigma_z\varphi}\paren{\sigma_x\cos\paren{\theta+\varphi}+\sigma_y\sin\paren{\theta+\varphi}}\\
  =&\paren{\cos\varphi-\ui\sigma_z\sin\varphi}\paren{\sigma_x\cos\paren{\theta+\varphi}+\sigma_y\sin\paren{\theta+\varphi}}\\
  =&\sigma_x\cos\varphi\cos\paren{\theta+\varphi}-\sigma_x\sin\varphi\sin\paren{\theta+\varphi}+\sigma_y\sin\varphi\cos\paren{\theta+\varphi}+\sigma_y\cos\varphi\sin\paren{\theta+\varphi}\\
  =&\sigma_x\cos\paren{\theta+2\varphi}+\sigma_y\sin\paren{\theta+2\varphi}
}
We can see that if we have $\omega'=-\dfrac{\delta_0}{2}$ and
$\phi'=-\dfrac{\phi_0}{2}$, we can cancel out the time dependent phase in the
Hamiltonian,
\eqar{
  U=&\exp\paren{-\frac{\ui\sigma_z}2\paren{\delta_0t+\phi_0}}\\
  \ui\partial_t|\psi'\rangle=&\paren{\Omega \sigma_x+\frac{\varepsilon-\delta_0}2\sigma_z}|\psi'\rangle
}
We can get the propagator in the rotatin frame by integrating this equation.
Let $\delta'\equiv\delta_0-\varepsilon$ be the real detuning (instead of
$\delta_0$, the detuning relative to some arbitrary reference energy),
\eqar{
  |\psi'\rangle_t=&\exp\paren{-\ui\paren{\Omega \sigma_x-\frac{\delta'}{2}\sigma_z}t}|\psi'\rangle_0\\
  =&\exp\paren{-\ui t\sqrt{\Omega^2+\frac{\delta'^2}{4}}\frac{2\Omega \sigma_x-\delta'\sigma_z}{\sqrt{4\Omega^2+\delta'^2}}}|\psi'\rangle_0
  \intertext{Let $\Omega'=\sqrt{\Omega^2+\dfrac{\delta'^2}{4}}$}
  |\psi'\rangle_t=&\exp\paren{-\ui\Omega't\frac{2\Omega\sigma_x-\delta'\sigma_z}{2\Omega'}}|\psi'\rangle_0\\
  =&\paren{\cos\Omega't-\ui\frac{2\Omega\sigma_x-\delta'\sigma_z}{2\Omega'}\sin\Omega't}|\psi'\rangle_0
}
This is our propagator in the rotating frame. We can now transform it back to
the original basis,
\eqar{
  |\psi\rangle_t=&U_t|\psi'\rangle_t\\
  =&U_t\paren{\cos\Omega't-\ui\frac{2\Omega\sigma_x-\delta'\sigma_z}{2\Omega'}\sin\Omega't}U_0^\dagger|\psi\rangle_0
  \intertext{The full time propagator $T=T_0T'$}
  T=&\exp\paren{-\frac{\ui\sigma_z}2\paren{\delta_0t+\phi_0}}\paren{\cos\Omega't-\ui\frac{2\Omega\sigma_x-\delta'\sigma_z}{2\Omega'}\sin\Omega't}\exp\paren{\frac{\ui\sigma_z}2\phi_0}\\
  =&\exp\paren{-\frac{\ui\delta_0t}2\sigma_z}\exp\paren{-\frac{\ui\sigma_z}2\phi_0}\paren{\cos\Omega't-\ui\frac{2\Omega\sigma_x-\delta'\sigma_z}{2\Omega'}\sin\Omega't}\exp\paren{\frac{\ui\sigma_z}2\phi_0}\\
  =&\exp\paren{-\frac{\ui\delta_0t}2\sigma_z}\paren{\cos\Omega't-\ui\frac{2\Omega\paren{\sigma_x\cos\phi_0+\sigma_y\sin\phi_0}-\delta'\sigma_z}{2\Omega'}\sin\Omega't}
  \intertext{And the effective time propagator we want,}
  T'=&T_0^\dagger T\\
  =&\exp\paren{-\frac{\ui\delta't}2\sigma_z}\paren{\cos\Omega't+\ui\frac{\delta'\sigma_z}{2\Omega'}\sin\Omega't-\ui\frac{\Omega}{\Omega'}\paren{\sigma_x\cos\phi_0+\sigma_y\sin\phi_0}\sin\Omega't}\\
  =&\begin{pmatrix}
    \exp\paren{-\ui\dfrac{\delta't}2}&0\\
    0&\exp\paren{\ui\dfrac{\delta't}2}
  \end{pmatrix}\begin{pmatrix}
    \cos\Omega't+\dfrac{\ui\delta'}{2\Omega'}\sin\Omega't&-\ui\dfrac{\Omega}{\Omega'}\ue^{-\ui\phi_0}\sin\Omega't\\
    -\ui\dfrac{\Omega}{\Omega'}\ue^{\ui\phi_0}\sin\Omega't&\cos\Omega't-\dfrac{\ui\delta'}{2\Omega'}\sin\Omega't
  \end{pmatrix}\\
  =&\begin{pmatrix}
    \exp\paren{-\ui\dfrac{\delta't}2}\paren{\cos\Omega't+\dfrac{\ui\delta'}{2\Omega'}\sin\Omega't}&-\ui\dfrac{\Omega}{\Omega'}\ue^{-\ui\phi_0}\exp\paren{-\ui\dfrac{\delta't}2}\sin\Omega't\\
    -\ui\dfrac{\Omega}{\Omega'}\ue^{\ui\phi_0}\exp\paren{\ui\dfrac{\delta't}2}\sin\Omega't&\exp\paren{\ui\dfrac{\delta't}2}\paren{\cos\Omega't-\dfrac{\ui\delta'}{2\Omega'}\sin\Omega't}
  \end{pmatrix}
}


\end{document}
