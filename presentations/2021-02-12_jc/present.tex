\documentclass[xcolor={table}]{beamer}
\mode<presentation>{
  \usetheme{Boadilla}
  \usefonttheme[onlylarge]{structurebold}
  \usefonttheme[stillsansseriflarge]{serif}
  \setbeamerfont*{frametitle}{size=\normalsize,series=\bfseries}
  % \setbeamertemplate{navigation symbols}{}
  \setbeamercovered{transparent}
}
\usepackage[english]{babel}
\usepackage[latin1]{inputenc}
\usepackage{times}
\usepackage[T1]{fontenc}
\usepackage{amsmath}
\usepackage{amssymb}
\usepackage{esint}
\usepackage{hyperref}
\usepackage{tikz}
\usepackage{xkeyval}
\usepackage{xargs}
\usepackage{verbatim}
\usepackage{listings}
\usepackage{multimedia}
\newcommand\hmmax{0}
\newcommand\bmmax{0}
\usepackage{bm}
\usepackage{siunitx}
\usepackage{xcolor,pifont}
\usepackage{upgreek}

\usetikzlibrary{
  arrows,
  calc,
  decorations.pathmorphing,
  decorations.pathreplacing,
  decorations.markings,
  fadings,
  positioning,
  shapes,
  arrows.meta
}
\usepgfmodule{oo}

\pgfdeclareradialshading{glow2}{\pgfpoint{0cm}{0cm}}{
  color(0mm)=(white);
  color(2mm)=(white);
  color(8mm)=(black);
  color(10mm)=(black)
}
\pgfdeclareradialshading{glow}{\pgfpoint{0cm}{0cm}}{
  color(0mm)=(white);
  color(5mm)=(white);
  color(9mm)=(black);
  color(10mm)=(black)
}

\begin{tikzfadingfrompicture}[name=glow fading]
  \shade [shading=glow] (0,0) circle (1);
\end{tikzfadingfrompicture}

\begin{tikzfadingfrompicture}[name=glow2 fading]
  \shade [shading=glow2] (0,0) circle (1);
\end{tikzfadingfrompicture}

\mode<handout>{
  \usepackage{pgfpages}
  \pgfpagesuselayout{4 on 1}[a4paper,landscape,border shrink=5mm]
  \setbeamercolor{background canvas}{bg=black!10}
}

\newcommand\pgfmathsinandcos[3]{%
  \pgfmathsetmacro#1{sin(#3)}%
  \pgfmathsetmacro#2{cos(#3)}%
}
\newcommand\LongitudePlane[3][current plane]{%
  \pgfmathsinandcos\sinEl\cosEl{#2} % elevation
  \pgfmathsinandcos\sint\cost{#3} % azimuth
  \tikzset{#1/.estyle={cm={\cost,\sint*\sinEl,0,\cosEl,(0,0)}}}
}
\newcommand\LatitudePlane[3][current plane]{%
  \pgfmathsinandcos\sinEl\cosEl{#2} % elevation
  \pgfmathsinandcos\sint\cost{#3} % latitude
  \pgfmathsetmacro\yshift{\cosEl*\sint}
  \tikzset{#1/.estyle={cm={\cost,0,0,\cost*\sinEl,(0,\yshift)}}} %
}
\newcommand\DrawLongitudeCircle[2][1]{
  \LongitudePlane{\angEl}{#2}
  \tikzset{current plane/.prefix style={scale=#1}}
  % angle of "visibility"
  \pgfmathsetmacro\angVis{atan(sin(#2)*cos(\angEl)/sin(\angEl))} %
  \draw[current plane] (\angVis:1) arc (\angVis:\angVis+180:1);
  \draw[current plane,dashed] (\angVis-180:1) arc (\angVis-180:\angVis:1);
}
\newcommand\DrawLatitudeCircleArrow[2][1]{
  \LatitudePlane{\angEl}{#2}
  \tikzset{current plane/.prefix style={scale=#1}}
  \pgfmathsetmacro\sinVis{sin(#2)/cos(#2)*sin(\angEl)/cos(\angEl)}
  % angle of "visibility"
  \pgfmathsetmacro\angVis{asin(min(1,max(\sinVis,-1)))}
  \draw[current plane,decoration={markings, mark=at position 0.6 with {\arrow{<}}},postaction={decorate},line width=.6mm] (\angVis:1) arc (\angVis:-\angVis-180:1);
  \draw[current plane,dashed,line width=.6mm] (180-\angVis:1) arc (180-\angVis:\angVis:1);
}
\newcommand\DrawLatitudeCircle[2][1]{
  \LatitudePlane{\angEl}{#2}
  \tikzset{current plane/.prefix style={scale=#1}}
  \pgfmathsetmacro\sinVis{sin(#2)/cos(#2)*sin(\angEl)/cos(\angEl)}
  % angle of "visibility"
  \pgfmathsetmacro\angVis{asin(min(1,max(\sinVis,-1)))}
  \draw[current plane] (\angVis:1) arc (\angVis:-\angVis-180:1);
  \draw[current plane,dashed] (180-\angVis:1) arc (180-\angVis:\angVis:1);
}
\newcommand\coil[1]{
  {\rh * cos(\t * pi r)}, {\apart * (2 * #1 + \t) + \rv * sin(\t * pi r)}
}
\makeatletter
\define@key{DrawFromCenter}{style}[{->}]{
  \tikzset{DrawFromCenterPlane/.style={#1}}
}
\define@key{DrawFromCenter}{r}[1]{
  \def\@R{#1}
}
\define@key{DrawFromCenter}{center}[(0, 0)]{
  \def\@Center{#1}
}
\define@key{DrawFromCenter}{theta}[0]{
  \def\@Theta{#1}
}
\define@key{DrawFromCenter}{phi}[0]{
  \def\@Phi{#1}
}
\presetkeys{DrawFromCenter}{style, r, center, theta, phi}{}
\newcommand*\DrawFromCenter[1][]{
  \setkeys{DrawFromCenter}{#1}{
    \pgfmathsinandcos\sint\cost{\@Theta}
    \pgfmathsinandcos\sinp\cosp{\@Phi}
    \pgfmathsinandcos\sinA\cosA{\angEl}
    \pgfmathsetmacro\DX{\@R*\cost*\cosp}
    \pgfmathsetmacro\DY{\@R*(\cost*\sinp*\sinA+\sint*\cosA)}
    \draw[DrawFromCenterPlane] \@Center -- ++(\DX, \DY);
  }
}
\newcommand*\DrawFromCenterText[2][]{
  \setkeys{DrawFromCenter}{#1}{
    \pgfmathsinandcos\sint\cost{\@Theta}
    \pgfmathsinandcos\sinp\cosp{\@Phi}
    \pgfmathsinandcos\sinA\cosA{\angEl}
    \pgfmathsetmacro\DX{\@R*\cost*\cosp}
    \pgfmathsetmacro\DY{\@R*(\cost*\sinp*\sinA+\sint*\cosA)}
    \draw[DrawFromCenterPlane] \@Center -- ++(\DX, \DY) node {#2};
  }
}
\makeatother

% not mandatory, but I though it was better to set it blank
\setbeamertemplate{headline}{}
\def\beamer@entrycode{\vspace{-\headheight}}

\tikzstyle{snakearrow} = [decorate, decoration={pre length=0.2cm,
  post length=0.2cm, snake, amplitude=.4mm,
  segment length=2mm},thick, ->]

%% document-wide tikz options and styles

\tikzset{%
  % >=latex, % option for nice arrows
  inner sep=0pt,%
  outer sep=2pt,%
  mark coordinate/.style={inner sep=0pt,outer sep=0pt,minimum size=3pt,
    fill=black,circle}%
}
\tikzset{
  % Define standard arrow tip
  >=stealth',
  % Define style for boxes
  punkt/.style={
    rectangle,
    rounded corners,
    draw=black, very thick,
    text width=8em,
    minimum height=2.5em,
    text centered},
}

\tikzset{onslide/.code args={<#1>#2}{%
    \only<#1>{\pgfkeysalso{#2}}
    % \pgfkeysalso doesn't change the path
  }}
\tikzset{alt/.code args={<#1>#2#3}{%
    \alt<#1>{\pgfkeysalso{#2}}{\pgfkeysalso{#3}}
    % \pgfkeysalso doesn't change the path
  }}
\tikzset{temporal/.code args={<#1>#2#3#4}{%
    \temporal<#1>{\pgfkeysalso{#2}}{\pgfkeysalso{#3}}{\pgfkeysalso{#4}}
    % \pgfkeysalso doesn't change the path
  }}

\makeatletter
\newbox\@backgroundblock
\newenvironment{backgroundblock}[2]{%
  \global\setbox\@backgroundblock=\vbox\bgroup%
  \unvbox\@backgroundblock%
  \vbox to0pt\bgroup\vskip#2\hbox to0pt\bgroup\hskip#1\relax%
}{\egroup\egroup\egroup}
\addtobeamertemplate{background}{\box\@backgroundblock}{}
\makeatother

\newcommand{\ud}{\mathrm{d}}
\newcommand{\ue}{\mathrm{e}}
\newcommand{\ui}{\mathrm{i}}
\newcommand{\Na}{\mathrm{Na}}
\newcommand{\Cs}{\mathrm{Cs}}
\newcommand{\abs}[1]{{\left|{#1}\right|}}
\newcommand{\paren}[1]{{\left({#1}\right)}}

% \def\timeleft{15:00->14:55}

\title[One-Way Quantum Computing]{A One-Way Quantum Computer}
\date{Feb. 15, 2021}
\author{Yichao Yu}
\institute{Ni Group}

\ifpdf
% Ensure reproducible output
\pdfinfoomitdate=1
\pdfsuppressptexinfo=-1
\pdftrailerid{}
\hypersetup{
  pdfcreator={},
  pdfproducer={}
}
\fi

\begin{document}

\begin{frame}{}
  \titlepage
\end{frame}

\begin{frame}{}
  \begin{center}
    \begin{columns}
      \column{9cm}
      \begin{itemize}
      \item 1D Cluster state
        \begin{itemize}
        \item Generation
        \item Properties
        \end{itemize}
      \item High dimensional cluster state
      \item Quantum circuit
      \item Gates and single qubit operations
      \end{itemize}
    \end{columns}
  \end{center}
\end{frame}

\begin{frame}[t]{1D Cluster State}
  \begin{center}
    \begin{tikzpicture}
      \draw[white,opacity=0] (-6, 0) -- (6, 0);

      \draw[line width=1,dashed] (3, 2) -- (4, 2);
      \draw[line width=1,dashed] (-3, 2) -- (-4, 2);
      \draw[line width=1] (-3, 2) -- (3, 2);
      \fill[red] (-3, 2) circle (0.15cm);
      \fill[red] (-1.5, 2) circle (0.15cm) node[black,above=0.2] {$a$};
      \fill[red] (0, 2) circle (0.15cm) node[black,above=0.2] {$a'$};
      \fill[red] (1.5, 2) circle (0.15cm);
      \fill[red] (3, 2) circle (0.15cm);

      \node at (-2.5, 0) {$\displaystyle H=\sum_{a,a'\in\Gamma}\frac{1+\sigma_z^{(a)}}{2}\frac{1-\sigma_z^{(a')}}{2}$};
      \node at (-2.5, -1) {$\displaystyle \Gamma=\left\{(a,a')|a'=a+1\right\}$};
      \node at (-2.5, -2.3) {$\displaystyle \mathcal{S}=\ue^{\ui\pi H}$};

      \visible<2->{
        \node at (2.5, 0) {$\displaystyle H=\sum\begin{pmatrix}0&&&\\&0&&\\&&1&\\&&&0\\\end{pmatrix}$};
        \node at (2.5, -3) {$\displaystyle \mathcal{S}=\bigotimes\begin{pmatrix}1&&&\\&1&&\\&&-1&\\&&&1\\\end{pmatrix}$};
      }
    \end{tikzpicture}
  \end{center}
\end{frame}

\begin{frame}[t]{1D Cluster State}
  \begin{center}
    \begin{tikzpicture}
      \draw[white,opacity=0] (-6, 0) -- (6, 0);

      \draw[line width=1,dashed] (3, 2) -- (4, 2);
      \draw[line width=1,dashed] (-3, 2) -- (-4, 2);
      \draw[line width=1] (-3, 2) -- (3, 2);
      \fill[red] (-3, 2) circle (0.15cm);
      \fill[red] (-1.5, 2) circle (0.15cm) node[black,above=0.2] {$a$};
      \fill[red] (0, 2) circle (0.15cm) node[black,above=0.2] {$a'$};
      \fill[red] (1.5, 2) circle (0.15cm);
      \fill[red] (3, 2) circle (0.15cm);
      \node at (0, 0) {$\displaystyle |\phi_N\rangle=\mathcal{S}\bigotimes_{a}|+\rangle_a
        =\frac{1}{2^{N/2}}\bigotimes_{a}\paren{|0\rangle_a\sigma_z^{a+1}+|1\rangle_a}$};
      \visible<2->{
        \node at (0, -1) {\footnotesize $\displaystyle |\phi_2\rangle=\frac{1}{\sqrt{2}}\paren{|0-\rangle+|1+\rangle}$};
      }
      \visible<3->{
        \node at (0, -1.8) {\footnotesize $\displaystyle |\phi_3\rangle=\frac{1}{\sqrt{2}}\paren{|+0-\rangle-|-1+\rangle}$};
      }
      \visible<4->{
        \node at (0, -2.6) {\footnotesize $\displaystyle |\phi_4\rangle=\frac{1}{2}\paren{|0-0-\rangle-|0+1+\rangle+|1+0-\rangle-|1-1+\rangle}$};
      }
      \visible<5->{
        \node at (0, -4) {\footnotesize $\displaystyle |\mathrm{GHZ}_N\rangle=\frac{1}{\sqrt{2}}\paren{\bigotimes_{a}|0\rangle_a+\bigotimes_{a}|1\rangle_a}$};
      }
    \end{tikzpicture}
  \end{center}
\end{frame}

\begin{frame}{1D Cluster State}
  \begin{center}
    \begin{tikzpicture}
      \node[text width=8.5cm,align=center] at (0, 0) {
        \begin{itemize}
        \item Maximum connectedness\\
          Ability to create Bell state by local measurements.\\
          \uncover<2->{Yes for both GHZ state and cluster state.}
        \item<3-> Persistency\\
          Minimum local measurements to destroy all entanglements.\\
          \uncover<-2,4->{GHZ: $P_e=1$, cluster: $P_e=\lfloor N/2\rfloor$}
        \end{itemize}
      };
    \end{tikzpicture}
  \end{center}
\end{frame}

\begin{frame}[t]{High Dimensional Cluster State}
  \begin{center}
    \begin{tikzpicture}
      \draw[line width=1] (-1, 3.5) -- (1, 3.5);
      \draw[line width=1] (-1, 2.5) -- (1, 2.5);
      \draw[line width=1] (-1, 1.5) -- (1, 1.5);
      \draw[line width=1,dashed] (-1.5, 3.5) -- (1.5, 3.5);
      \draw[line width=1,dashed] (-1.5, 2.5) -- (1.5, 2.5);
      \draw[line width=1,dashed] (-1.5, 1.5) -- (1.5, 1.5);

      \draw[line width=1] (-1, 3.5) -- (-1, 1.5);
      \draw[line width=1] (0, 3.5) -- (0, 1.5);
      \draw[line width=1] (1, 3.5) -- (1, 1.5);
      \draw[line width=1,dashed] (-1, 4) -- (-1, 1);
      \draw[line width=1,dashed] (0, 4) -- (0, 1);
      \draw[line width=1,dashed] (1, 4) -- (1, 1);

      \fill[red] (-1, 3.5) circle (0.15cm);
      \fill[red] (0, 3.5) circle (0.15cm);
      \fill[red] (1, 3.5) circle (0.15cm);
      \fill[red] (-1, 2.5) circle (0.15cm) node[black,above right=0.2] {$a'$};
      \fill[red] (0, 2.5) circle (0.15cm);
      \fill[red] (1, 2.5) circle (0.15cm);
      \fill[red] (-1, 1.5) circle (0.15cm) node[black,above right=0.2] {$a$};
      \fill[red] (0, 1.5) circle (0.15cm) node[black,above right=0.2] {$a'$};
      \fill[red] (1, 1.5) circle (0.15cm);
      \node at (0, -0.5) {$\displaystyle H=\sum_{a,a'\in\Gamma}\frac{1+\sigma_z^{(a)}}{2}\frac{1-\sigma_z^{(a')}}{2}$};
      \node at (0, -1.5) {$\displaystyle \Gamma=\left\{(a,a')|a'=a+\hat e_i\right\}$};
    \end{tikzpicture}
  \end{center}
\end{frame}

\begin{frame}[t]{High Dimensional Cluster State}
  \begin{center}
    \begin{tikzpicture}
      \node at (0, 0) {$\displaystyle H=\sum_{a,a'\in\Gamma}\frac{1+\sigma_z^{(a)}}{2}\frac{1-\sigma_z^{(a')}}{2}$};
      \node at (0, -1) {$\displaystyle [H, \sigma_z^{(a)}]=0,\ \ [\mathcal{S}, \sigma_z^{(a)}]=0$};

      \visible<2->{
        \begin{scope}[shift={(-2, -3)}, scale=0.5]
          \draw[line width=1] (-2, 2) -- (2, 2);
          \draw[line width=1] (-2, 1) -- (2, 1);
          \draw[line width=1] (-2, 0) -- (2, 0);
          \draw[line width=1] (-2, -1) -- (2, -1);
          \draw[line width=1] (-2, -2) -- (2, -2);

          \draw[line width=1] (-2, 2) -- (-2, -2);
          \draw[line width=1] (-1, 2) -- (-1, -2);
          \draw[line width=1] (0, 2) -- (0, -2);
          \draw[line width=1] (1, 2) -- (1, -2);
          \draw[line width=1] (2, 2) -- (2, -2);

          \fill[red] (-2, 2) circle (0.15cm);
          \fill[red] (-1, 2) circle (0.15cm);
          \fill[red] (0, 2) circle (0.15cm);
          \fill[red] (1, 2) circle (0.15cm);
          \fill[red] (2, 2) circle (0.15cm);
          \fill[red] (-2, 1) circle (0.15cm);
          \fill[red] (-1, 1) circle (0.15cm);
          \fill[red] (0, 1) circle (0.15cm);
          \fill[red] (1, 1) circle (0.15cm);
          \fill[red] (2, 1) circle (0.15cm);
          \fill[red] (-2, 0) circle (0.15cm);
          \fill[red] (-1, 0) circle (0.15cm);
          \fill[red] (0, 0) circle (0.15cm);
          \fill[red] (1, 0) circle (0.15cm);
          \fill[red] (2, 0) circle (0.15cm);
          \fill[red] (-2, -1) circle (0.15cm);
          \fill[red] (-1, -1) circle (0.15cm);
          \fill[red] (0, -1) circle (0.15cm);
          \fill[red] (1, -1) circle (0.15cm);
          \fill[red] (2, -1) circle (0.15cm);
          \fill[red] (-2, -2) circle (0.15cm);
          \fill[red] (-1, -2) circle (0.15cm);
          \fill[red] (0, -2) circle (0.15cm);
          \fill[red] (1, -2) circle (0.15cm);
          \fill[red] (2, -2) circle (0.15cm);
        \end{scope}

        \draw[->,>=stealth,line width=3] (-0.7, -3) -- node[above] {\Large $\sigma_z$} (0.7, -3);

        \begin{scope}[shift={(2, -3)}, scale=0.5]
          \draw[line width=1] (-2, 2) -- (-1, 2);
          \draw[line width=1] (1, 2) -- (2, 2);
          \draw[line width=1] (0, 1) -- (2, 1);
          \draw[line width=1] (-2, 0) -- (-1, 0);
          \draw[line width=1] (1, 0) -- (2, 0);
          \draw[line width=1] (-2, -1) -- (1, -1);
          \draw[line width=1] (1, -2) -- (2, -2);

          \draw[line width=1] (-2, 2) -- (-2, -2);
          \draw[line width=1] (-1, 0) -- (-1, -1);
          \draw[line width=1] (1, 2) -- (1, -2);
          \draw[line width=1] (2, 2) -- (2, 0);

          \fill[red] (-2, 2) circle (0.15cm);
          \fill[red] (-1, 2) circle (0.15cm);
          \draw[red,fill=white] (0, 2) circle (0.15cm);
          \fill[red] (1, 2) circle (0.15cm);
          \fill[red] (2, 2) circle (0.15cm);
          \fill[red] (-2, 1) circle (0.15cm);
          \draw[red,fill=white] (-1, 1) circle (0.15cm);
          \fill[red] (0, 1) circle (0.15cm);
          \fill[red] (1, 1) circle (0.15cm);
          \fill[red] (2, 1) circle (0.15cm);
          \fill[red] (-2, 0) circle (0.15cm);
          \fill[red] (-1, 0) circle (0.15cm);
          \draw[red,fill=white] (0, 0) circle (0.15cm);
          \fill[red] (1, 0) circle (0.15cm);
          \fill[red] (2, 0) circle (0.15cm);
          \fill[red] (-2, -1) circle (0.15cm);
          \fill[red] (-1, -1) circle (0.15cm);
          \fill[red] (0, -1) circle (0.15cm);
          \fill[red] (1, -1) circle (0.15cm);
          \draw[red,fill=white] (2, -1) circle (0.15cm);
          \fill[red] (-2, -2) circle (0.15cm);
          \draw[red,fill=white] (-1, -2) circle (0.15cm);
          \draw[red,fill=white] (0, -2) circle (0.15cm);
          \fill[red] (1, -2) circle (0.15cm);
          \fill[red] (2, -2) circle (0.15cm);
        \end{scope}
      }

      \visible<3->{
        \begin{scope}[shift={(0, -5.5)}, scale=0.9]
          \draw[line width=1] (-1, 1) -- (0, 1);
          \draw[line width=1] (-1, 0) -- (0, 0);
          \draw[line width=1] (-1, -1) -- (0, -1);

          \draw[line width=1] (-1, 1) -- (-1, -1);
          \draw[line width=1] (0, 1) -- (0, -1);

          \fill[red] (-1, 1) circle (0.15cm);
          \fill[red] (0, 1) circle (0.15cm);
          \fill[red] (-1, 0) circle (0.15cm);
          \fill[red] (0, 0) circle (0.15cm) node[below left=0.1] {\small $a$};
          \draw[blue,line width=1,fill=white] (1, 0) circle (0.15cm)
          node[below left=0.1] {\small $b$}
          node[right=0.25] {\small $\sigma^{(b)}_z=s$};
          \fill[red] (-1, -1) circle (0.15cm);
          \fill[red] (0, -1) circle (0.15cm);

          \draw[blue,->,>=stealth,line width=1.5,shorten <=0.15cm,shorten >=0.1cm]
          (1, 0) to[bend right] node[above] {$\sigma_z^s$} (0, 0);
        \end{scope}
      }
    \end{tikzpicture}
  \end{center}
\end{frame}

\begin{frame}[t]{High Dimensional Cluster State}
  \begin{center}
    \begin{tikzpicture}
      \draw[line width=1] (-1, 3.5) -- (1, 3.5);
      \draw[line width=1] (-1, 2.5) -- (1, 2.5);
      \draw[line width=1] (-1, 1.5) -- (1, 1.5);

      \draw[line width=1] (-1, 3.5) -- (-1, 1.5);
      \draw[line width=1] (0, 3.5) -- (0, 1.5);
      \draw[line width=1] (1, 3.5) -- (1, 1.5);

      \fill[red] (-1, 3.5) circle (0.15cm);
      \fill[red] (0, 3.5) circle (0.15cm) node[black,above right=0.2] {$a'$};
      \fill[red] (1, 3.5) circle (0.15cm);
      \fill[red] (-1, 2.5) circle (0.15cm) node[black,above right=0.2] {$a'$};
      \fill[red] (0, 2.5) circle (0.15cm) node[black,above right=0.2] {$a$};
      \fill[red] (1, 2.5) circle (0.15cm) node[black,above right=0.2] {$a'$};
      \fill[red] (-1, 1.5) circle (0.15cm);
      \fill[red] (0, 1.5) circle (0.15cm) node[black,above right=0.2] {$a'$};
      \fill[red] (1, 1.5) circle (0.15cm);
      \node at (0, 0) {$\displaystyle K_a=\sigma_x^{(a)}\bigotimes_{a'\in\Gamma'}\sigma_z^{(a')}$};
      \node at (0, -1)
      {\footnotesize $\displaystyle \Gamma'=\left\{(a,a')|a'=a\pm\hat e_i\right\}$};

      \visible<2->{
        \node[align=center,text width=6cm] at (6, 1.7) {
          \begin{itemize}
          \item<2->$K_a|\phi_N\rangle=\pm|\phi_N\rangle$
          \item<3->$\{K_a,\sigma_z^{(a)}\}=0$
          \item<4->$[K_a,K_b]=0$
          \item<5->$\left.[K_a,\sigma_z^{(b)}]\right|_{a\neq b}=0$
          \item<6->Independent
          \item<7->Complete
          \item<8->Equivalent definitions of cluster state
            \begin{itemize}
            \item<-7,9-> Apply full $\mathcal{S}$ and measure $\sigma_z$'s on removed sites
            \item<-7,10-> Apply partial $\mathcal{S}$ and apply $\sigma_z$'s on remaining sites
            \item<-7,11-> Eigenstates of all $K_a$
            \end{itemize}
          \end{itemize}
        };
      }
    \end{tikzpicture}
  \end{center}
\end{frame}

\begin{frame}{Quantum Circuit on Cluster State}
  \begin{center}
    \begin{tikzpicture}
      \node[draw,minimum height=2cm,minimum width=0.4cm,rounded corners=.2cm] at (-3, 0) {};
      \node[above] at (-3, 1) {\footnotesize $|\psi_{\rm in}\rangle_1$};
      \node[draw,minimum height=2cm,minimum width=1.5cm,rounded corners=.2cm,align=center]
      at (-1.5, 0) {$U_1$\\\scriptsize $|\psi_U\rangle_1$};
      \node[draw,minimum height=2cm,minimum width=0.4cm,rounded corners=.2cm] at (0, 0) {};
      \node[above] at (0, 1)
      {\footnotesize \only<-2>{$|\psi_{\rm out}\rangle_1$}
        \only<3->{$|\psi_{\rm out}\rangle_1\!\!=\!|\psi_{\rm in}\rangle_2$}};
      \visible<3->{
        \node[draw,minimum height=2cm,minimum width=1.5cm,rounded corners=.2cm,align=center]
        at (1.5, 0) {$U_2$\\\scriptsize $|\psi_U\rangle_2$};
        \node[draw,minimum height=2cm,minimum width=0.4cm,rounded corners=.2cm] at (3, 0) {};
        \node[above] at (3, 1) {\footnotesize $|\psi_{\rm out}\rangle_2$};
      }
      \draw (-2.8, 0) -- (-2.25, 0);
      \draw (-0.75, 0) -- (-0.2, 0);
      \draw (-2.8, -0.7) -- (-2.25, -0.7);
      \draw (-0.75, -0.7) -- (-0.2, -0.7);
      \draw (-2.8, 0.7) -- (-2.25, 0.7);
      \draw (-0.75, 0.7) -- (-0.2, 0.7);

      \visible<3->{
        \draw (0.2, 0) -- (0.75, 0);
        \draw (2.25, 0) -- (2.8, 0);
        \draw (0.2, -0.7) -- (0.75, -0.7);
        \draw (2.25, -0.7) -- (2.8, -0.7);
        \draw (0.2, 0.7) -- (0.75, 0.7);
        \draw (2.25, 0.7) -- (2.8, 0.7);
      }

      \visible<2->{
        \node[draw,blue,rounded corners=.1cm,inner sep = 0.08cm,align=center]
        at (-4.3, -2) {Apply\\$\mathcal{S}_1$};
        \node[draw,red,rounded corners=.1cm,inner sep = 0.08cm,align=center]
        at (-2.75, -2) {Measure\\$|\psi_{\rm in}\rangle_1$};
        \node[draw,red,rounded corners=.1cm,inner sep = 0.08cm,align=center]
        at (-1, -2) {Measure\\$|\psi_{U}\rangle_1$};

        \draw[dashed,line width=1] (-3.3, -0.9) -- (-4.8, -1.5);
        \draw[dashed,line width=1] (-0.7, -0.9) -- (-0.3, -1.5);
      }

      \visible<3->{
        \node[draw,blue,rounded corners=.1cm,inner sep = 0.08cm,align=center]
        at (0.55, -2) {Apply\\$\mathcal{S}_2$};
        \node[draw,red,rounded corners=.1cm,inner sep = 0.08cm,align=center]
        at (2.1, -2) {Measure\\$|\psi_{\rm in}\rangle_2$};
        \node[draw,red,rounded corners=.1cm,inner sep = 0.08cm,align=center]
        at (3.85, -2) {Measure\\$|\psi_{U}\rangle_2$};

        \draw[dashed,line width=1] (-0.2, -0.9) -- (0.0, -1.5);
        \draw[dashed,line width=1] (2.3, -0.9) -- (4.55, -1.5);
      }

      \visible<4->{
        \node[draw,blue,rounded corners=.1cm,inner sep = 0.08cm,align=center]
        at (-4.3, -4) {Apply\\$\mathcal{S}_1$};
        \node[draw,blue,rounded corners=.1cm,inner sep = 0.08cm,align=center]
        at (-2.95, -4) {Apply\\$\mathcal{S}_2$};

        \node[draw,red,rounded corners=.1cm,inner sep = 0.08cm,align=center]
        at (-1.4, -4) {Measure\\$|\psi_{\rm in}\rangle_1$};
        \node[draw,red,rounded corners=.1cm,inner sep = 0.08cm,align=center]
        at (0.35, -4) {Measure\\$|\psi_{U}\rangle_1$};
        \node[draw,red,rounded corners=.1cm,inner sep = 0.08cm,align=center]
        at (2.1, -4) {Measure\\$|\psi_{\rm in}\rangle_2$};
        \node[draw,red,rounded corners=.1cm,inner sep = 0.08cm,align=center]
        at (3.85, -4) {Measure\\$|\psi_{U}\rangle_2$};
      }
    \end{tikzpicture}
  \end{center}
\end{frame}

\begin{frame}{Gates and Single Qubit Operations (Single Qubit)}
  \begin{center}
    \begin{tikzpicture}
      \node[align=left,text width=9cm] at (0, 0) {
        Propagation: Measure $\sigma_x$
        {
          \visible<4->{
            \begin{align*}
              \mathcal{S}\paren{a|0\rangle_1+b|1\rangle_1}|+\rangle_2
              =&|+\rangle_1\paren{a|-\rangle_2+b|+\rangle_2}\\
               &+|-\rangle_1\paren{a|-\rangle_2-b|+\rangle_2}
            \end{align*}
          }
        }
      };
      \begin{scope}[shift={(3.1, 1.8)}]
        \visible<2->{
          \draw[line width=1] (-1, 0) -- node[above,blue] {\small $\mathcal{S}$}
          node[below=0.5cm,blue!80!black] {\small Step 1} (0, 0);
          \fill[red] (-1, 0) circle (0.2cm) node[white] {\footnotesize $\psi_i$};
          \fill[red] (0, 0) circle (0.2cm);
        }

        \visible<3->{
          \draw[line width=1] (0.8, 0) --
          node[below=0.5cm,blue!80!black] {\small Step 2} (1.8, 0);
          \fill[red] (0.8, 0) circle (0.2cm)
          node[above=0.21cm,blue,align=center,
          execute at begin node=\setlength{\baselineskip}{7pt}]
          {\footnotesize Measure\\\small $\sigma_x$};
          \fill[red] (1.8, 0) circle (0.2cm) node[white] {\footnotesize $\psi_o$};
        }
      \end{scope}

      \visible<5->{
        \node[draw,minimum height=1cm,minimum width=1.7cm,rounded corners=.1cm] (RG)
        at (-1, -2) {$\sigma_x/\!\!-\!\!\ui\sigma_y$};
        \node[draw,minimum height=1cm,minimum width=1cm,rounded corners=.1cm] (HG)
        at (1, -2) {$H$};
        \draw[line width=0.7] (-2.5, -2) -- (-1.85, -2);
        \draw[line width=0.7] (-0.15, -2) -- (0.5, -2);
        \draw[line width=0.7] (1.5, -2) -- (2.15, -2);
      }
      \visible<6->{
        \node[align=left,text width=9cm] at (0, -4) {
          Rotation: Measure $\sigma_x\cos\theta+\sigma_y\sin\theta$
        };
      }
    \end{tikzpicture}
  \end{center}
\end{frame}

\begin{frame}{Gates and Single Qubit Operations (CNOT)}
  \begin{center}
    \begin{tikzpicture}
      \begin{scope}[shift={(-4, 3)}]
        \draw[line width=1] (-1, 0) -- node[blue,above] {\small $\mathcal{S}$} (0, 0)
        -- node[blue,above] {\small $\mathcal{S}$} (1, 0);
        \draw[line width=1] (0, 0) -- node[blue,right] {\small $\mathcal{S}$} (0, -1);
        \fill[red] (-1, 0) circle (0.17cm) node[white] {\scriptsize $\psi_i$}
        node[black,left=0.2] {\scriptsize target};
        \fill[red] (0, 0) circle (0.17cm);
        \fill[red] (1, 0) circle (0.17cm);
        \fill[red] (0, -1) circle (0.17cm)
        node[black,left=0.2] {\scriptsize control};
      \end{scope}
      \begin{scope}[shift={(0, 3)}]
        \draw[line width=1] (-1, 0) -- (0, 0) -- (1, 0);
        \draw[line width=1] (0, 0) -- (0, -1);
        \fill[red] (-1, 0) circle (0.17cm)
        node[above=0.2cm,blue,align=center,
        execute at begin node=\setlength{\baselineskip}{7pt}]
        {\footnotesize Measure\\\small $\sigma_x$};
        \fill[red] (0, 0) circle (0.17cm);
        \fill[red] (1, 0) circle (0.17cm);
        \fill[red] (0, -1) circle (0.17cm);
      \end{scope}
      \begin{scope}[shift={(4, 3)}]
        \draw[line width=1] (-1, 0) -- (0, 0) -- (1, 0);
        \draw[line width=1] (0, 0) -- (0, -1);
        \fill[red] (-1, 0) circle (0.17cm);
        \fill[red] (0, 0) circle (0.17cm)
        node[above=0.2cm,blue,align=center,
        execute at begin node=\setlength{\baselineskip}{7pt}]
        {\footnotesize Measure\\\small $\sigma_x$};
        \fill[red] (1, 0) circle (0.17cm) node[white] {\scriptsize $\psi_o$};
        \fill[red] (0, -1) circle (0.17cm);
      \end{scope}

      \draw[->,>=stealth,line width=2] (-5.5, 1.5) -- (5.5, 1.5);

      \visible<2->{
        \begin{scope}[shift={(-4.2, 0)}]
          \draw[line width=1] (-1, 0) -- node[blue,above] {\small $\mathcal{S}$} (0, 0) -- (1, 0);
          \draw[line width=1] (0, 0) -- (0, -1);
          \fill[red] (-1, 0) circle (0.17cm) node[white] {$z$}
          node[above=0.2cm,blue,align=center,
          execute at begin node=\setlength{\baselineskip}{7pt}]
          {\footnotesize Measure\\\small $\sigma_x$};
          \fill[blue] (0, 0) circle (0.17cm) node[white] {$x$};
          \fill[red] (1, 0) circle (0.17cm);
          \fill[red] (0, -1) circle (0.17cm) node[white] {$z$};
        \end{scope}
      }

      \visible<3->{
        \begin{scope}[shift={(-0.5, 0)}]
          \draw[line width=1] (0, 0) -- (1, 0);
          \draw[line width=1] (0, 0) --
          node[blue,right] {\small $\mathcal{S}\!=\!\text{C-}\sigma_z$} (0, -1);
          \draw[red,fill=white] (-1, 0) circle (0.17cm);
          \fill[blue] (0, 0) circle (0.17cm) node[white] {$x$};
          \fill[red] (1, 0) circle (0.17cm);
          \fill[red] (0, -1) circle (0.17cm) node[white] {$z$};

          \node at (0, -2.5) {\footnotesize $\displaystyle \mathcal{S}=\bigotimes\begin{pmatrix}1&&&\\&1&&\\&&-1&\\&&&1\\\end{pmatrix}$};
        \end{scope}
      }

      \visible<4->{
        \begin{scope}[shift={(3.2, 0)}]
          \draw[line width=1] (0, 0) --
          node[blue,above] {\small $\mathcal{S}$} (1, 0);
          \draw[line width=1] (0, 0) -- (0, -1);
          \draw[red,fill=white] (-1, 0) circle (0.17cm);
          \fill[blue] (0, 0) circle (0.17cm) node[white] {\small $x'$}
          node[above=0.2cm,blue,align=center,
          execute at begin node=\setlength{\baselineskip}{7pt}]
          {\footnotesize Measure\\\small $\sigma_x$};
          \fill[red] (1, 0) circle (0.17cm) node[white] {\small $z'$};
          \fill[red] (0, -1) circle (0.17cm) node[white] {$z$};

          \draw[red,dotted,->,>=stealth,line width=1.5,shorten <=0.17cm,shorten >=0.17cm]
          (0, -1) to[bend right=50]
          node[below right,pos=0.2] {\footnotesize C-$\sigma_x$, i.e. CNOT} (1, 0);
        \end{scope}
      }
    \end{tikzpicture}
  \end{center}
\end{frame}

\begin{frame}{Questions?}
  \begin{center}
  \end{center}
\end{frame}

\end{document}
