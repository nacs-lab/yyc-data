\documentclass{beamer}
\mode<presentation>{
  \usetheme{Boadilla}
  \usefonttheme[onlylarge]{structurebold}
  \usefonttheme[stillsansseriflarge]{serif}
  \setbeamerfont*{frametitle}{size=\normalsize,series=\bfseries}
  % \setbeamertemplate{navigation symbols}{}
  \setbeamercovered{transparent}
}
\usepackage[english]{babel}
\usepackage[latin1]{inputenc}
\usepackage{times}
\usepackage[T1]{fontenc}
\usepackage{amsmath}
\usepackage{amssymb}
\usepackage{esint}
\usepackage{hyperref}
\usepackage{tikz}
\usepackage{xkeyval}
\usepackage{xargs}
\usepackage{verbatim}
\usepackage{listings}
\usepackage{multimedia}
\newcommand\hmmax{0}
\newcommand\bmmax{0}
\usepackage{bm}
\usepackage{siunitx}
\usetikzlibrary{
  arrows,
  calc,
  decorations.pathmorphing,
  decorations.pathreplacing,
  decorations.markings,
  fadings,
  positioning,
  shapes,
  arrows.meta
}
\usepgfmodule{oo}

\pgfdeclareradialshading{glow2}{\pgfpoint{0cm}{0cm}}{
  color(0mm)=(white);
  color(2mm)=(white);
  color(8mm)=(black);
  color(10mm)=(black)
}
\pgfdeclareradialshading{glow}{\pgfpoint{0cm}{0cm}}{
  color(0mm)=(white);
  color(5mm)=(white);
  color(9mm)=(black);
  color(10mm)=(black)
}

\begin{tikzfadingfrompicture}[name=glow fading]
  \shade [shading=glow] (0,0) circle (1);
\end{tikzfadingfrompicture}

\begin{tikzfadingfrompicture}[name=glow2 fading]
  \shade [shading=glow2] (0,0) circle (1);
\end{tikzfadingfrompicture}

\mode<handout>{
  \usepackage{pgfpages}
  \pgfpagesuselayout{4 on 1}[a4paper,landscape,border shrink=5mm]
  \setbeamercolor{background canvas}{bg=black!10}
}

\newcommand\pgfmathsinandcos[3]{%
  \pgfmathsetmacro#1{sin(#3)}%
  \pgfmathsetmacro#2{cos(#3)}%
}
\newcommand\LongitudePlane[3][current plane]{%
  \pgfmathsinandcos\sinEl\cosEl{#2} % elevation
  \pgfmathsinandcos\sint\cost{#3} % azimuth
  \tikzset{#1/.estyle={cm={\cost,\sint*\sinEl,0,\cosEl,(0,0)}}}
}
\newcommand\LatitudePlane[3][current plane]{%
  \pgfmathsinandcos\sinEl\cosEl{#2} % elevation
  \pgfmathsinandcos\sint\cost{#3} % latitude
  \pgfmathsetmacro\yshift{\cosEl*\sint}
  \tikzset{#1/.estyle={cm={\cost,0,0,\cost*\sinEl,(0,\yshift)}}} %
}
\newcommand\DrawLongitudeCircle[2][1]{
  \LongitudePlane{\angEl}{#2}
  \tikzset{current plane/.prefix style={scale=#1}}
  % angle of "visibility"
  \pgfmathsetmacro\angVis{atan(sin(#2)*cos(\angEl)/sin(\angEl))} %
  \draw[current plane] (\angVis:1) arc (\angVis:\angVis+180:1);
  \draw[current plane,dashed] (\angVis-180:1) arc (\angVis-180:\angVis:1);
}
\newcommand\DrawLatitudeCircleArrow[2][1]{
  \LatitudePlane{\angEl}{#2}
  \tikzset{current plane/.prefix style={scale=#1}}
  \pgfmathsetmacro\sinVis{sin(#2)/cos(#2)*sin(\angEl)/cos(\angEl)}
  % angle of "visibility"
  \pgfmathsetmacro\angVis{asin(min(1,max(\sinVis,-1)))}
  \draw[current plane,decoration={markings, mark=at position 0.6 with {\arrow{<}}},postaction={decorate},line width=.6mm] (\angVis:1) arc (\angVis:-\angVis-180:1);
  \draw[current plane,dashed,line width=.6mm] (180-\angVis:1) arc (180-\angVis:\angVis:1);
}
\newcommand\DrawLatitudeCircle[2][1]{
  \LatitudePlane{\angEl}{#2}
  \tikzset{current plane/.prefix style={scale=#1}}
  \pgfmathsetmacro\sinVis{sin(#2)/cos(#2)*sin(\angEl)/cos(\angEl)}
  % angle of "visibility"
  \pgfmathsetmacro\angVis{asin(min(1,max(\sinVis,-1)))}
  \draw[current plane] (\angVis:1) arc (\angVis:-\angVis-180:1);
  \draw[current plane,dashed] (180-\angVis:1) arc (180-\angVis:\angVis:1);
}
\newcommand\coil[1]{
  {\rh * cos(\t * pi r)}, {\apart * (2 * #1 + \t) + \rv * sin(\t * pi r)}
}
\makeatletter
\define@key{DrawFromCenter}{style}[{->}]{
  \tikzset{DrawFromCenterPlane/.style={#1}}
}
\define@key{DrawFromCenter}{r}[1]{
  \def\@R{#1}
}
\define@key{DrawFromCenter}{center}[(0, 0)]{
  \def\@Center{#1}
}
\define@key{DrawFromCenter}{theta}[0]{
  \def\@Theta{#1}
}
\define@key{DrawFromCenter}{phi}[0]{
  \def\@Phi{#1}
}
\presetkeys{DrawFromCenter}{style, r, center, theta, phi}{}
\newcommand*\DrawFromCenter[1][]{
  \setkeys{DrawFromCenter}{#1}{
    \pgfmathsinandcos\sint\cost{\@Theta}
    \pgfmathsinandcos\sinp\cosp{\@Phi}
    \pgfmathsinandcos\sinA\cosA{\angEl}
    \pgfmathsetmacro\DX{\@R*\cost*\cosp}
    \pgfmathsetmacro\DY{\@R*(\cost*\sinp*\sinA+\sint*\cosA)}
    \draw[DrawFromCenterPlane] \@Center -- ++(\DX, \DY);
  }
}
\newcommand*\DrawFromCenterText[2][]{
  \setkeys{DrawFromCenter}{#1}{
    \pgfmathsinandcos\sint\cost{\@Theta}
    \pgfmathsinandcos\sinp\cosp{\@Phi}
    \pgfmathsinandcos\sinA\cosA{\angEl}
    \pgfmathsetmacro\DX{\@R*\cost*\cosp}
    \pgfmathsetmacro\DY{\@R*(\cost*\sinp*\sinA+\sint*\cosA)}
    \draw[DrawFromCenterPlane] \@Center -- ++(\DX, \DY) node {#2};
  }
}
\makeatother

% not mandatory, but I though it was better to set it blank
\setbeamertemplate{headline}{}
\def\beamer@entrycode{\vspace{-\headheight}}

\tikzstyle{snakearrow} = [decorate, decoration={pre length=0.2cm,
  post length=0.2cm, snake, amplitude=.4mm,
  segment length=2mm},thick, ->]

%% document-wide tikz options and styles

\tikzset{%
  % >=latex, % option for nice arrows
  inner sep=0pt,%
  outer sep=2pt,%
  mark coordinate/.style={inner sep=0pt,outer sep=0pt,minimum size=3pt,
    fill=black,circle}%
}
\tikzset{
  % Define standard arrow tip
  >=stealth',
  % Define style for boxes
  punkt/.style={
    rectangle,
    rounded corners,
    draw=black, very thick,
    text width=8em,
    minimum height=2.5em,
    text centered},
}

\tikzset{onslide/.code args={<#1>#2}{%
    \only<#1>{\pgfkeysalso{#2}}
    % \pgfkeysalso doesn't change the path
  }}
\tikzset{alt/.code args={<#1>#2#3}{%
    \alt<#1>{\pgfkeysalso{#2}}{\pgfkeysalso{#3}}
    % \pgfkeysalso doesn't change the path
  }}
\tikzset{temporal/.code args={<#1>#2#3#4}{%
    \temporal<#1>{\pgfkeysalso{#2}}{\pgfkeysalso{#3}}{\pgfkeysalso{#4}}
    % \pgfkeysalso doesn't change the path
  }}

\makeatletter
\newbox\@backgroundblock
\newenvironment{backgroundblock}[2]{%
  \global\setbox\@backgroundblock=\vbox\bgroup%
  \unvbox\@backgroundblock%
  \vbox to0pt\bgroup\vskip#2\hbox to0pt\bgroup\hskip#1\relax%
}{\egroup\egroup\egroup}
\addtobeamertemplate{background}{\box\@backgroundblock}{}
\makeatother

% \def\timeleft{15:00->14:55}

\title{NaCs lab update}
\date{Feb. 21, 2020}
\author{Yichao Yu}
\institute{Ni Group}

\begin{document}

\pgfdeclarelayer{tweezer}
\pgfsetlayers{tweezer,main}
\pgfooclass{tweezer}{
  \method tweezer() {
  }
  \method drawTweezer(#1,#2,#3) {
    \begin{pgfonlayer}{tweezer}
      \shade[shading=radial,path fading=glow fading,shift={(#1,#2)},rotate=90,yscale=1,
      fill opacity=0.9,inner color=#3]
      plot[draw,samples=200,domain=-1.5:1.5] function {sqrt(0.01 + x**2 / 5)}
      -- plot[draw,samples=200,domain=1.5:-1.5] function {-sqrt(0.01 + x**2 / 5)};
    \end{pgfonlayer}
  }
  \method drawAtom(#1,#2,#3,#4) {
    \fill [#4,path fading=glow2 fading] (#1,#2) circle (#3);
  }
  \method drawNaAtom(#1,#2,#3) {
    \pgfoothis.drawAtom(#1,#2,#3,orange);
  }
  \method drawCsAtom(#1,#2,#3) {
    \pgfoothis.drawAtom(#1,#2,#3,blue);
  }
  \method drawNaTweezer(#1,#2) {
    \pgfoothis.drawTweezer(#1,#2,orange!35!black!30);
  }
  \method drawCsTweezer(#1,#2) {
    \pgfoothis.drawTweezer(#1,#2,blue!30!black!30);
  }
  \method up(#1,#2) {
    \pgfoothis.drawCsTweezer(#1,#2);
    \pgfoothis.drawNaAtom(#1,#2+0.06,0.12);
    \pgfoothis.drawCsAtom(#1,#2-0.06,0.16);
  }
  \method down(#1,#2) {
    \pgfoothis.drawCsTweezer(#1,#2);
    \pgfoothis.drawCsAtom(#1,#2+0.06,0.16);
    \pgfoothis.drawNaAtom(#1,#2-0.06,0.12);
  }
  \method naTrap(#1,#2) {
    \pgfoothis.drawNaTweezer(#1,#2);
    \pgfoothis.drawNaAtom(#1,#2,0.12);
  }
  \method csTrap(#1,#2) {
    \pgfoothis.drawCsTweezer(#1,#2);
    \pgfoothis.drawCsAtom(#1,#2,0.16);
  }
}
\pgfoonew \mytweezer=new tweezer()

{
  \usebackgroundtemplate{
    \makebox[\paperwidth][c]{\centering\includegraphics[width=\paperwidth]{front_bg.png}}
  }
  \setbeamercolor{title}{fg=cyan!50}
  \setbeamercolor{author}{fg=white}
  \setbeamercolor{institute}{fg=white}
  \setbeamercolor{date}{fg=white}
  \begin{frame}{}
    \titlepage
  \end{frame}
}

% Raman transfer
%% Transfer to bound state with a Raman process.
%% No Rabi oscillation last time with basically this exact slide
%% Unfortunately this time still no coherence transfer and just as sad.
%% Since then, we have basically been in a cycle between trying to understand the issue
%% and finding a better approach to avoid an issue.
%% As said, no oscillatiton yet. However, did have more understanding of our problems and
%% that's what I want to talk about today.
\begin{frame}{}
  \begin{center}
    \begin{tikzpicture}
      \begin{scope}[scale=0.7]
        \draw[->,line width=1.2] (0, 0) -- (0, 8);
        \node[above,rotate=90] at (0, 4) {Energy};
        \draw[->,line width=1.2] (0, 0) -- (8, 0);
        \node[below] at (4, -0.5) {Internuclear distance};

        \draw[cyan!85!blue] (1.0269 + 0.25, 2.5) -- (7.25, 2.5);
        \draw[cyan!85!blue] (1.0793 + 0.25, -0.4631 + 2.5) -- (4.5 + 0.25, -0.4631 + 2.5);

        \draw[line width=1.1,cyan!85!blue]
        plot[samples=200,domain=1:7,variable=\x]
        ({\x + 0.25}, {6.8*\x^(-3.4)-6.5*\x^(-1.7) + 2.5});
        \node[cyan!85!blue] at (3.75, 1.0) {$a^3\Sigma^+$};

        \draw[line width=1.1,red]
        plot[samples=200,domain=1:7.5,variable=\x]
        ({\x - 0.75}, {9.2*\x^(-2.5)-9.0*\x^(-1.3) + 7.5});
        \node[above right,red] at (0.55, 7.2) {$c^3\Sigma^+$};

        \mytweezer.drawNaAtom(6.55, 2.6, 0.12)
        \mytweezer.drawCsAtom(7.05, 2.6, 0.10)

        \mytweezer.drawNaAtom(4.08, 2.15, 0.12)
        \mytweezer.drawCsAtom(4.25, 2.15, 0.10)

        \draw[black!40,dashed,line width=1] (0.25, 6) -- (7.25, 6);

        \draw[->,green!80!black,line width=0.8] (6.8, 2.7) -- (5.5, 6)
        node[midway, right, align=left] {Tweezer};
        \draw[->,green!80!black,line width=0.8] (5.45, 6) -- (4.165, 2.25);
      \end{scope}
      \visible<3->{
        \node[below right,align=center,red] at (6.5, 6) {\textbf{Still}};
      }
      \visible<2->{
        \node[below right,align=center] at (6.5, 5.5) {
          No Rabi oscillation\\
          \\
          \includegraphics[width=2.5cm]{crying_face.png}};
      }
      \visible<4->{
        \node[below right,align=center,text width=5cm] at (6.5, 1.5) {
          \begin{itemize}
          \item Understand the issue
          \item Find a better approach
          \end{itemize}
        };
      }
    \end{tikzpicture}
  \end{center}
\end{frame}

% What's wrong.
%% Previous talks we talked about some of the suspicions.
%% I'm going to repeat some of them with the hope that I can present the most relevant
%% and informative one in a logical order rather than chronological order.
%% We also tried multiple pathways as mentioned and I'll generally talk about
%% the shared properties except few special cases.
%%
%% Need a better matric (than the absence of Rabi oscillation).
%% Binary data is not informative especially when the data is all `0`.
%%
%% Condition: Linewidth < Rabi rate
%%
%% * It has to be related to a single line.
%%   % (Detuning figure)
%% * Fluctuation
%%   % This one we actually couldn't rule out still, but all the direct measurement of
%%   % of the intensity noise suggests it shouldn't be too bad.
%%   % Also spoiler from our 1.5 on FB molecule that they have shorter lifetime than expected
%%   % as well.
%% * Scattering
%%   % How many photon.
\begin{frame}{What can go wrong?}
  \begin{columns}
    \column{5cm}
    \begin{center}
      \only<2>{
        \begin{tabular}{|c|c|}
          \hline
          Condition&Rabi Oscillation\\\hline
          b&No\\\hline
          r&No\\\hline
          K&No\\\hline
          No&No\\\hline
          $\cdots$&Still No\\\hline
          $\cdots$&$\cdots$\\\hline
        \end{tabular}
      }
      \only<3->{
        \only<3>{
          \[ \frac{\Gamma_{\text{(Line width)}}}{\Omega_{\text{(Rabi frequency)}}} \]
        }
        \only<4->{
          \[ \frac{\color{red}\Gamma_{\text{(Line width)}}}{\Omega_{\text{(Rabi frequency)}}} \]
        }
        \visible<5->{
          \begin{block}{}
            \begin{itemize}
            \item<5-> Single PA line effect
            \item<6-> Fluctuation
            \item<7-> Scattering
            \end{itemize}
          \end{block}
        }
      }
    \end{center}

    \column{7cm}
    \vspace{-0.7cm}
    \begin{center}
      \begin{tikzpicture}
        \visible<-4,7->{
          \begin{scope}[scale=0.6]
            \draw[->,line width=1.2] (0, 0) -- (0, 8);
            \node[above,rotate=90] at (0, 4) {Energy};
            \draw[->,line width=1.2] (0, 0) -- (8, 0);
            \node[below] at (4, -0.5) {Internuclear distance};

            \draw[cyan!85!blue] (1.0269 + 0.25, 2.5) -- (7.25, 2.5);
            \draw[cyan!85!blue] (1.0793 + 0.25, -0.4631 + 2.5) -- (4.5 + 0.25, -0.4631 + 2.5);

            \draw[line width=1.1,cyan!85!blue]
            plot[samples=200,domain=1:7,variable=\x]
            ({\x + 0.25}, {6.8*\x^(-3.4)-6.5*\x^(-1.7) + 2.5});
            \node[cyan!85!blue] at (3.75, 1.0) {$a^3\Sigma^+$};

            \draw[line width=1.1,red]
            plot[samples=200,domain=1:7.5,variable=\x]
            ({\x - 0.75}, {9.2*\x^(-2.5)-9.0*\x^(-1.3) + 7.5});
            \node[above right,red] at (0.55, 7.2) {$c^3\Sigma^+$};

            \mytweezer.drawNaAtom(6.55, 2.6, 0.12)
            \mytweezer.drawCsAtom(7.05, 2.6, 0.10)

            \mytweezer.drawNaAtom(4.08, 2.15, 0.12)
            \mytweezer.drawCsAtom(4.25, 2.15, 0.10)

            \draw[black!40,dashed,line width=1] (0.25, 6) -- (7.25, 6);

            \draw[->,green!80!black,line width=0.8] (6.8, 2.7) -- (5.5, 6);
            \draw[->,green!80!black,line width=0.8] (5.45, 6) -- (4.165, 2.25);
          \end{scope}
        }
        \visible<5>{
          \node[right] at (0, 1.8) {
            \includegraphics[width=5cm]{../../experiments/nacs_202002/imgs/data_20200220_raman_vs_freq_3311}};
        }
        \visible<6>{
          \node[right,align=center] at (0, 1.8) {
            Feshbach molecule lifetime\\
            \includegraphics[width=5cm]{fb_lifetime}
          };
        }
        \visible<8->{
          \node[right,red] at (0.6, -1.5) {
            How many photons?
          };
        }
      \end{tikzpicture}
    \end{center}
  \end{columns}
\end{frame}

% Two photons
%% * We have stronger tweezer power
%% * Hard to predict
%% * Evidence from other group
%% We were talking about these for a while and even found some evidence for it
%% Excited state
%% * Frequency dependency
%% * Power dependency
%% So two photon effect is definately there.
%% Before we realized two things.
%% * Excited state effect is different from ground state effect.
%% * Linear scaling
\begin{frame}{Two photon scattering}
  \vspace{-1cm}
  \begin{columns}
    \column{5cm}
    \only<-5>{
      \begin{block}{}
        \begin{itemize}
        \item<1-> Stronger intensity than bulk gas
        \item<2-> Less accurate/no prediction
        \item<3-> Evidence from other group
        \end{itemize}
      \end{block}
    }
    \only<6->{
      \begin{center}
        {\Large \textit{Until \dots}}
        \begin{block}{}
          \begin{itemize}
          \item<7-> Two photon up vs down
          \item<8-> One beam vs two beams
          \item<9-> Linear dependency on power for Raman line width
          \end{itemize}
        \end{block}
      \end{center}
    }
    \column{6cm}
    \begin{center}
      \begin{tikzpicture}
        \visible<4-6>{
          \node[above] at (0, 0) {
            \includegraphics[width=5cm]{../2019-09-05_itamp/pa_widths}
          };
        }
        \visible<5-6>{
          \node[below] at (0, 0) {
            \includegraphics[width=4.5cm]{../../experiments/pa_201908/imgs/data_20190819_pa_linewidths}
          };
        }
        \visible<7-8>{
          \node at (0, 1) {
            \includegraphics[width=5cm]{../../notes/two-photon-broadening/kenneth_figure}
          };
        }
        \visible<9->{
          \node at (0, 1) {
            \includegraphics[width=5.5cm]{../../experiments/nacs_202001/imgs/data_20200103_raman_width_vs_power_3311_2}
          };
        }
      \end{tikzpicture}
    \end{center}
  \end{columns}
\end{frame}

% One photon
%% * Direct contradiction
%% * Should be ``easy''
%%   % I got three consecutive days where someone told me because it's single photon
%%   % it should be easy to understand.
%% * Do calculation
%%   % SO coupling?
%%   % Coupling to motional decoherence?
\begin{frame}{One photon scattering (i.e. the easy kind)}
  \begin{center}
    \begin{tikzpicture}
      \visible<-8>{
        \draw[line width=1] (-5, -2) -- (-3, -2);
        \draw[line width=1] (-3.5, 4) -- (-1.5, 4);
        \draw[dashed,line width=0.7] (-3.5, 3.5) -- (-1.5, 3.5);
        \draw[->,line width=1,green!80!black] (-2.55, 3.5) -- (-4, -2)
        node[midway,left=0.1] {$\Omega$};
        \draw[<->,line width=0.7] (-3.3, 3.5) -- (-3.3, 4);
        \node[left,purple] at (-3.6, 3.75) {$\Delta$};
        \draw[snakearrow,line width=1,blue!80!black] (-2, 4) -- (-1, 2.5)
        node[midway,right=0.1] {$\Gamma_e$};
      }
      \begin{scope}[shift={(0, 1.2)}]
        \visible<2-6>{
          \node[right,align=left] at (1, 1) {
            $\gamma_{scatter}=\dfrac{{\color{blue!80!black} \Gamma_e}\ {\color{green!80!black} \Omega}^2}{4\ {\color{purple} \Delta}^2}$};
        }
        \visible<7->{
          \node[right,align=left] at (1, 1) {
            $\gamma_{scatter}=\dfrac{{\color{blue!80!black} \Gamma_e}\ {\color{green!80!black} \Omega}^2}{4\ {\color{purple} \Delta}^2}{\color{red}\times 10 \sim 100}$};
        }
        \visible<3->{
          \draw[->, line width=1] (1, 0.9) -- (0.5, 0.5) node[below] {\footnotesize Raman linewidth};
        }
        \visible<4->{
          \draw[->, line width=1,blue!80!black] (2.6, 1.4) -- (1.4, 2.4) node[above] {\footnotesize PA linewidth};
        }
        \visible<5->{
          \draw[->, line width=1,green!80!black] (3.25, 1.4) -- (4.2, 2.6) node[above] {\footnotesize From light shift};
        }
        \visible<6->{
          \draw[->, line width=1,purple] (3.14, 0.5) -- (2.9, -0.2) node[below] {\footnotesize Detuning};
        }
        \visible<8->{
          \node[below,align=center,text width=5cm] at (2.9, -1) {
            \begin{block}{}
              \begin{itemize}
              \item<8-> Spin-orbit coupling?
              \item<9-> Decay into motional continuum?
              \end{itemize}
            \end{block}
          };
        }
        \visible<9->{
          \begin{scope}[scale=0.6, shift={(-9, -4)}]
            \draw[->,line width=1.2] (0, 0) -- (0, 8);
            \node[above,rotate=90] at (0, 4) {Energy};
            \draw[->,line width=1.2] (0, 0) -- (8, 0);
            \node[below] at (4, -0.5) {Internuclear distance};

            \draw[cyan!85!blue] (1.0793 + 0.25, -0.4631 + 2.5) -- (4.5 + 0.25, -0.4631 + 2.5);
            \draw[line width=1.1,cyan!85!blue]
            plot[samples=200,domain=1:7,variable=\x]
            ({\x + 0.25}, {6.8*\x^(-3.4)-6.5*\x^(-1.7) + 2.5});
            \node[cyan!85!blue] at (3.75, 1.0) {$a^3\Sigma^+$};

            \draw[red] (0.509, 6) -- (2.406, 6);
            \draw[line width=1.1,red]
            plot[samples=200,domain=1:7.5,variable=\x]
            ({\x - 0.75}, {9.2*\x^(-2.5)-9.0*\x^(-1.3) + 7.5});
            \node[above right,red] at (0.55, 7.2) {$c^3\Sigma^+$};

            \mytweezer.drawNaAtom(2.38, 2.15, 0.12)
            \mytweezer.drawCsAtom(2.55, 2.15, 0.10)
            \draw[snakearrow,green!80!black,line width=0.8] (1.358, 6) -- (2.46, 2.25)
            node[midway,left] {\only<11->{$20\%$}};

            \visible<10->{
              \draw[->,orange] (4.55, 3.6) -- +(-0.4, 0);
              \draw[->,blue] (5.05, 3.6) -- +(0.4, 0);
              \mytweezer.drawNaAtom(4.55, 3.6, 0.12)
              \mytweezer.drawCsAtom(5.05, 3.6, 0.10)
              \draw[snakearrow,green!80!black,line width=0.8] (1.458, 6) -- (4.8, 3.7)
              node[midway,above right] {\only<11->{$80\%$}};
            }
          \end{scope}
        }
      \end{scope}
    \end{tikzpicture}
  \end{center}
\end{frame}

% What's next.
%% Back to the cycle
%% * Understand our issue
%%   % Calculations
%% * Make it work
%%   % Find a better state (33+22)
\begin{frame}{Next}
  \begin{center}
    \begin{itemize}
    \item Understand the issue\\
      \uncover<2->{
        Calculating decay rate due to different effects.
      }
    \item Find a better approach\\
      \uncover<3->{
        New initial state: $F^{Cs}=3, m_{F}^{Cs}=3; F^{Na}=2, m_{F}^{Na}=2$\\
        \visible<3->{Thanks Ben \& ACME.}
      }
    \end{itemize}
  \end{center}
\end{frame}

\end{document}
