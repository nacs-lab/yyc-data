\documentclass{beamer}
\mode<presentation>{
  \usetheme{Boadilla}
  \usefonttheme[onlylarge]{structurebold}
  \usefonttheme[stillsansseriflarge]{serif}
  \setbeamerfont*{frametitle}{size=\normalsize,series=\bfseries}
  % \setbeamertemplate{navigation symbols}{}
  \setbeamercovered{transparent}
}
\usepackage[english]{babel}
\usepackage[latin1]{inputenc}
\usepackage{times}
\usepackage[T1]{fontenc}
\usepackage{amsmath}
\usepackage{amssymb}
\usepackage{esint}
\usepackage{hyperref}
\usepackage{tikz}
\usepackage{xkeyval}
\usepackage{xargs}
\usepackage{xcolor}
\usepackage{verbatim}
\usepackage{listings}
\usepackage{multimedia}
\usepackage{bm}
\usepackage{siunitx}
\usetikzlibrary{
  arrows,
  calc,
  decorations.pathmorphing,
  decorations.pathreplacing,
  decorations.markings,
  fadings,
  positioning,
  shapes,
  arrows.meta
}
\usepgfmodule{oo}

\pgfdeclareradialshading{glow2}{\pgfpoint{0cm}{0cm}}{
  color(0mm)=(white);
  color(2mm)=(white);
  color(8mm)=(black);
  color(10mm)=(black)
}
\pgfdeclareradialshading{glow}{\pgfpoint{0cm}{0cm}}{
  color(0mm)=(white);
  color(5mm)=(white);
  color(9mm)=(black);
  color(10mm)=(black)
}

\begin{tikzfadingfrompicture}[name=glow fading]
  \shade [shading=glow] (0,0) circle (1);
\end{tikzfadingfrompicture}

\begin{tikzfadingfrompicture}[name=glow2 fading]
  \shade [shading=glow2] (0,0) circle (1);
\end{tikzfadingfrompicture}

\mode<handout>{
  \usepackage{pgfpages}
  \pgfpagesuselayout{4 on 1}[a4paper,landscape,border shrink=5mm]
  \setbeamercolor{background canvas}{bg=black!10}
}

\newcommand\pgfmathsinandcos[3]{%
  \pgfmathsetmacro#1{sin(#3)}%
  \pgfmathsetmacro#2{cos(#3)}%
}
\newcommand\LongitudePlane[3][current plane]{%
  \pgfmathsinandcos\sinEl\cosEl{#2} % elevation
  \pgfmathsinandcos\sint\cost{#3} % azimuth
  \tikzset{#1/.estyle={cm={\cost,\sint*\sinEl,0,\cosEl,(0,0)}}}
}
\newcommand\LatitudePlane[3][current plane]{%
  \pgfmathsinandcos\sinEl\cosEl{#2} % elevation
  \pgfmathsinandcos\sint\cost{#3} % latitude
  \pgfmathsetmacro\yshift{\cosEl*\sint}
  \tikzset{#1/.estyle={cm={\cost,0,0,\cost*\sinEl,(0,\yshift)}}} %
}
\newcommand\DrawLongitudeCircle[2][1]{
  \LongitudePlane{\angEl}{#2}
  \tikzset{current plane/.prefix style={scale=#1}}
  % angle of "visibility"
  \pgfmathsetmacro\angVis{atan(sin(#2)*cos(\angEl)/sin(\angEl))} %
  \draw[current plane] (\angVis:1) arc (\angVis:\angVis+180:1);
  \draw[current plane,dashed] (\angVis-180:1) arc (\angVis-180:\angVis:1);
}
\newcommand\DrawLatitudeCircleArrow[2][1]{
  \LatitudePlane{\angEl}{#2}
  \tikzset{current plane/.prefix style={scale=#1}}
  \pgfmathsetmacro\sinVis{sin(#2)/cos(#2)*sin(\angEl)/cos(\angEl)}
  % angle of "visibility"
  \pgfmathsetmacro\angVis{asin(min(1,max(\sinVis,-1)))}
  \draw[current plane,decoration={markings, mark=at position 0.6 with {\arrow{<}}},postaction={decorate},line width=.6mm] (\angVis:1) arc (\angVis:-\angVis-180:1);
  \draw[current plane,dashed,line width=.6mm] (180-\angVis:1) arc (180-\angVis:\angVis:1);
}
\newcommand\DrawLatitudeCircle[2][1]{
  \LatitudePlane{\angEl}{#2}
  \tikzset{current plane/.prefix style={scale=#1}}
  \pgfmathsetmacro\sinVis{sin(#2)/cos(#2)*sin(\angEl)/cos(\angEl)}
  % angle of "visibility"
  \pgfmathsetmacro\angVis{asin(min(1,max(\sinVis,-1)))}
  \draw[current plane] (\angVis:1) arc (\angVis:-\angVis-180:1);
  \draw[current plane,dashed] (180-\angVis:1) arc (180-\angVis:\angVis:1);
}
\newcommand\coil[1]{
  {\rh * cos(\t * pi r)}, {\apart * (2 * #1 + \t) + \rv * sin(\t * pi r)}
}
\makeatletter
\define@key{DrawFromCenter}{style}[{->}]{
  \tikzset{DrawFromCenterPlane/.style={#1}}
}
\define@key{DrawFromCenter}{r}[1]{
  \def\@R{#1}
}
\define@key{DrawFromCenter}{center}[(0, 0)]{
  \def\@Center{#1}
}
\define@key{DrawFromCenter}{theta}[0]{
  \def\@Theta{#1}
}
\define@key{DrawFromCenter}{phi}[0]{
  \def\@Phi{#1}
}
\presetkeys{DrawFromCenter}{style, r, center, theta, phi}{}
\newcommand*\DrawFromCenter[1][]{
  \setkeys{DrawFromCenter}{#1}{
    \pgfmathsinandcos\sint\cost{\@Theta}
    \pgfmathsinandcos\sinp\cosp{\@Phi}
    \pgfmathsinandcos\sinA\cosA{\angEl}
    \pgfmathsetmacro\DX{\@R*\cost*\cosp}
    \pgfmathsetmacro\DY{\@R*(\cost*\sinp*\sinA+\sint*\cosA)}
    \draw[DrawFromCenterPlane] \@Center -- ++(\DX, \DY);
  }
}
\newcommand*\DrawFromCenterText[2][]{
  \setkeys{DrawFromCenter}{#1}{
    \pgfmathsinandcos\sint\cost{\@Theta}
    \pgfmathsinandcos\sinp\cosp{\@Phi}
    \pgfmathsinandcos\sinA\cosA{\angEl}
    \pgfmathsetmacro\DX{\@R*\cost*\cosp}
    \pgfmathsetmacro\DY{\@R*(\cost*\sinp*\sinA+\sint*\cosA)}
    \draw[DrawFromCenterPlane] \@Center -- ++(\DX, \DY) node {#2};
  }
}
\makeatother

% not mandatory, but I though it was better to set it blank
\setbeamertemplate{headline}{}
\def\beamer@entrycode{\vspace{-\headheight}}

\tikzstyle{snakearrow} = [decorate, decoration={pre length=0.2cm,
  post length=0.2cm, snake, amplitude=.4mm,
  segment length=2mm},thick, ->]

%% document-wide tikz options and styles

\tikzset{%
  % >=latex, % option for nice arrows
  inner sep=0pt,%
  outer sep=2pt,%
  mark coordinate/.style={inner sep=0pt,outer sep=0pt,minimum size=3pt,
    fill=black,circle}%
}
\tikzset{
  % Define standard arrow tip
  >=stealth',
  % Define style for boxes
  punkt/.style={
    rectangle,
    rounded corners,
    draw=black, very thick,
    text width=8em,
    minimum height=2.5em,
    text centered},
}

\tikzset{onslide/.code args={<#1>#2}{%
    \only<#1>{\pgfkeysalso{#2}}
    % \pgfkeysalso doesn't change the path
  }}
\tikzset{alt/.code args={<#1>#2#3}{%
    \alt<#1>{\pgfkeysalso{#2}}{\pgfkeysalso{#3}}
    % \pgfkeysalso doesn't change the path
  }}
\tikzset{temporal/.code args={<#1>#2#3#4}{%
    \temporal<#1>{\pgfkeysalso{#2}}{\pgfkeysalso{#3}}{\pgfkeysalso{#4}}
    % \pgfkeysalso doesn't change the path
  }}

\makeatletter
\newbox\@backgroundblock
\newenvironment{backgroundblock}[2]{%
  \global\setbox\@backgroundblock=\vbox\bgroup%
  \unvbox\@backgroundblock%
  \vbox to0pt\bgroup\vskip#2\hbox to0pt\bgroup\hskip#1\relax%
}{\egroup\egroup\egroup}
\addtobeamertemplate{background}{\box\@backgroundblock}{}
\makeatother

\ifpdf
% Ensure reproducible output
\pdfinfoomitdate=1
\pdfsuppressptexinfo=-1
\pdftrailerid{}
\fi

\title{Building Single Molecules from Single Atoms}
\date{Jul. 2020}
\author{Yichao Yu}
\institute{Ni Group/Harvard}

%%% Outline

%% Goal
% New system and tool for quantum computing and quantum simulation
% Requires quantum system that is
% * Fully controllable
% * Allow entanglement
% * Scalable (to tens or hundreds)

%% System
% Many other systems in AMO
% Each approaches all have their advantages and disadvantages.
% These are the properties that we exploid and the reasons that we pick our system.
% Just like other systems they have their difficulties and that'll be the main focus of my talk.

% Molecules
% * Dipole intraction
% %% Strong, tunable, helps with achieving fast interaction
% %% while maintaining long coherence time.
% * Rich internal states (e.g. qubit state selection)

% Tweezers
% * Single site detection
% * Single site manipulation

% Two approaches (Cooling molecules vs forming molecules)
% (given our goal of combining optical tweezer with ultracold molecules)
% * Cooling molecules (show CaF picture)
% %% Difficulty on achieving lower temperature
% * Assembly (....)
% %% Difficulty on forming the molecule
% %% Allow us to use more existing technologies
% %% Transfer of coherent control

% Designed by taking advantage of the new techniques that are developed over the past decades

%% Experiment design

% * MOT
% * Tweezer trapping (metion switching) (Na + Cs single atom picture)
% * RSC in tweezer (extremely common in ion traps, also realized for atoms in tweezers before)
% * Merging
% * Forming molecules

%% Lab pictures

%% Atom preparation
% (since our goal is to transfer atom coherence to molecule coherence)
% RSC, common in ion trap, but needs tweak to work for us.
% Cs RSC, following ion trap procedure (advice from Till Roseband to sweep the pulse time)
% Na RSC, not as simple.
% High L-D parameter, high initial temperature
% The real challenge in the coupling dead zone.
% Needs higher order sideband.
% Simulation guided sequence.

%% Molecule formation

% Two step transfer scheme

% Interaction shift?

% Raman transition
% (DAMOP talk)

% Conclusion
% * Experiment aiming to create flexible array of ultracold molecules for applications like ...
% * Atomic state preparation (internal and external, and how we overcome the challenge for Na)
% * Molecular state transfer (coherence, maintaining control, all optical)
% * Next step: transfer to ground state (could be from FB molecule).

\begin{document}

\pgfdeclarelayer{tweezer}
\pgfsetlayers{tweezer,main}
\pgfooclass{tweezer}{
  \method tweezer() {
  }
  \method drawTweezer(#1,#2,#3) {
    \begin{pgfonlayer}{tweezer}
      \shade[shading=radial,path fading=glow fading,shift={(#1,#2)},rotate=90,yscale=1,
      fill opacity=0.9,inner color=#3]
      plot[draw,samples=200,domain=-1.5:1.5] function {sqrt(0.01 + x**2 / 5)}
      -- plot[draw,samples=200,domain=1.5:-1.5] function {-sqrt(0.01 + x**2 / 5)};
    \end{pgfonlayer}
  }
  \method drawAtom(#1,#2,#3,#4) {
    \fill [#4,path fading=glow2 fading] (#1,#2) circle (#3);
  }
  \method drawNaAtom(#1,#2,#3) {
    \pgfoothis.drawAtom(#1,#2,#3,orange);
  }
  \method drawCsAtom(#1,#2,#3) {
    \pgfoothis.drawAtom(#1,#2,#3,blue);
  }
  \method drawNaTweezer(#1,#2) {
    \pgfoothis.drawTweezer(#1,#2,orange!35!black!30);
  }
  \method drawCsTweezer(#1,#2) {
    \pgfoothis.drawTweezer(#1,#2,blue!30!black!30);
  }
  \method up(#1,#2) {
    \pgfoothis.drawCsTweezer(#1,#2);
    \pgfoothis.drawNaAtom(#1,#2+0.06,0.12);
    \pgfoothis.drawCsAtom(#1,#2-0.06,0.16);
  }
  \method down(#1,#2) {
    \pgfoothis.drawCsTweezer(#1,#2);
    \pgfoothis.drawCsAtom(#1,#2+0.06,0.16);
    \pgfoothis.drawNaAtom(#1,#2-0.06,0.12);
  }
  \method naTrap(#1,#2) {
    \pgfoothis.drawNaTweezer(#1,#2);
    \pgfoothis.drawNaAtom(#1,#2,0.12);
  }
  \method csTrap(#1,#2) {
    \pgfoothis.drawCsTweezer(#1,#2);
    \pgfoothis.drawCsAtom(#1,#2,0.16);
  }
}
\pgfoonew \mytweezer=new tweezer()

%% Title
% I'm Yichao from the Ni lab at Harvard and today I'll tell you about our experiment on
% creating single NaCs molecules in optical tweezers.
{
  \usebackgroundtemplate{
    \makebox[\paperwidth][c]{\centering\includegraphics[width=\paperwidth]{front_bg.png}}
  }
  \setbeamercolor{title}{fg=white}
  \setbeamercolor{author}{fg=white}
  \setbeamercolor{institute}{fg=white}
  \setbeamercolor{date}{fg=white}
  \begin{frame}{}
    \titlepage
  \end{frame}
}

%% Goal
% The long term goal of our experiment is to create a new system
% for quantum computing and quantum simulation, as well as, in general,
% adding new tools to the so-called AMO toolbox for potentially other applications.
% These applications require our system to be fully controllable to single particle level
% as well as allowing creating entanglement between the particles.

% There are many good candidates, and even just in AMO there are
% neutral atoms, ions, Rydbergs etc, so why are we adding yet another one to the list.
% Well, first of all, broadly speaking, you'll never know what you'll get unless you try it.
% There are often pleasant supprises, even from the difficulties, as you'll see later.
% And this really applies to many other things including some big discovery as well.
% More concretely, although some approaches have got closer to the goal then others,
% no one have actually got there yet and there are still big challeges for all of them.
% So the door is still open, and although I'm definately not going to claim we have the
% best system in the world, I do believe we are bringing new in new ideas and
% new solutions for some of the issues in other systems.
\begin{frame}{}
\end{frame}

%% System
% Hopefully I've now convinced you that despite all the achievements in other
% approaches, it is still interesting to study new systems, if it is different enough
% properties from the existing one. So now let's look at our system.

% The two properties that I've emphasized on initially are full quantum control and
% entanglement. And let's first talk about entanglement.
% Well, entangling different particles essentially means interaction.
% And this is the main reason we chose dipole molecules.
% Compared to neutral atoms, they have much stronger and longer range interactions,
% roughly `kHz` about a micron apart. This might not seem like much compared to other strongly
% interacting systems like ions and Rydberg atoms, the interacting molecular
% states are long lived and easily tunable, which is useful for reducing coupling with
% the enviroment and achieve long coherence time.
% Molecules also have a very rich internal state, which is the main reason for the tunability,
% but it also give us a lot of freedom in selection our qubit states, for example.

% The other important property we are aiming for is full quantum control,
% and for that our main tool is the optical tweezer.
% Since the tweezer is generally form through a high-NA objective,
% it very naturally gives us the resolution for single site detection and even manipulation
% of the molecules. This technique has been developped in many other groups for atoms including
% here in Cindy's lab, demostrating sideband cooling, near deterministic loading
% and rearranging and you'll see later in my talk how we use the tweezer to achieve
% quantum control on our molcules.
\begin{frame}{Molecules in tweezers}
\end{frame}

%%
% So since our goal is to create ultracold molecules, there are basically two approaches.
% First is to first get molecule and then get ultracold, in another word, direct cooling
% of hot molecules to ultracold temperature without destroying or otherwise loosing the molecule.
% This is a very hard problem because of the difficulties in laser cooling, but there
% has been a lot of progress in this area, including an experiment here at JILA in Jun's group.
% In fact, the Doyle group at Harvard have recently demostrated ultracold CaF molecules
% in an array of optical tweezers, as you can see in this picture.

% The other approach, which is what we use, is to get ultracold, and then get molecule.
% In another word, we would cool whatever make up the molecule, in our case atoms,
% before turning them into molecule coherently, without heating them, and in general maintaining
% the control we had on them in this process.

% Comparing the two approaches, the main challenge currently for the direct cooling approach
% is to cool and in general controlling the molecules in the tweezer. Whereas for us,
% our challenge is to first obtaining the control on the atom, and the coherence creation of the
% molecules. These two main challenges will be what I'm going to focus on
% for the remainder of my talk.
\begin{frame}{}
\end{frame}

% So here's an outline for the rest of my talk.
% I'll first give an overview of our experiment, both the procedure and the apparatus.
% Then I'll discuss about one example of achieving full control on our atoms.
% And after that I'll talk about how we create the molecule while maintaining this
% control, before giving my conclusion.
\begin{frame}{Outline}
  \tableofcontents
\end{frame}

\begin{frame}{}
\end{frame}

\begin{frame}{}
\end{frame}

%% Backup slides

\end{document}
