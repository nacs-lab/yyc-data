\documentclass{beamer}
\mode<presentation>{
  \usetheme{Boadilla}
  \usefonttheme[onlylarge]{structurebold}
  \usefonttheme[stillsansseriflarge]{serif}
  \setbeamerfont*{frametitle}{size=\normalsize,series=\bfseries}
  % \setbeamertemplate{navigation symbols}{}
  \setbeamercovered{transparent}
}
\usepackage[english]{babel}
\usepackage[latin1]{inputenc}
\usepackage{times}
\usepackage[T1]{fontenc}
\usepackage{amsmath}
\usepackage{amssymb}
\usepackage{esint}
\usepackage{hyperref}
\usepackage{tikz}
\usepackage{xkeyval}
\usepackage{xargs}
\usepackage{xcolor}
\usepackage{verbatim}
\usepackage{listings}
\usepackage{multimedia}
\usepackage{bm}
\usepackage{siunitx}
\usetikzlibrary{
  arrows,
  calc,
  decorations.pathmorphing,
  decorations.pathreplacing,
  decorations.markings,
  fadings,
  positioning,
  shapes,
  arrows.meta
}
\usepgfmodule{oo}

\pgfdeclareradialshading{glow2}{\pgfpoint{0cm}{0cm}}{
  color(0mm)=(white);
  color(2mm)=(white);
  color(8mm)=(black);
  color(10mm)=(black)
}
\pgfdeclareradialshading{glow}{\pgfpoint{0cm}{0cm}}{
  color(0mm)=(white);
  color(5mm)=(white);
  color(9mm)=(black);
  color(10mm)=(black)
}

\begin{tikzfadingfrompicture}[name=glow fading]
  \shade [shading=glow] (0,0) circle (1);
\end{tikzfadingfrompicture}

\begin{tikzfadingfrompicture}[name=glow2 fading]
  \shade [shading=glow2] (0,0) circle (1);
\end{tikzfadingfrompicture}

\mode<handout>{
  \usepackage{pgfpages}
  \pgfpagesuselayout{4 on 1}[a4paper,landscape,border shrink=5mm]
  \setbeamercolor{background canvas}{bg=black!10}
}

\newcommand\pgfmathsinandcos[3]{%
  \pgfmathsetmacro#1{sin(#3)}%
  \pgfmathsetmacro#2{cos(#3)}%
}
\newcommand\LongitudePlane[3][current plane]{%
  \pgfmathsinandcos\sinEl\cosEl{#2} % elevation
  \pgfmathsinandcos\sint\cost{#3} % azimuth
  \tikzset{#1/.estyle={cm={\cost,\sint*\sinEl,0,\cosEl,(0,0)}}}
}
\newcommand\LatitudePlane[3][current plane]{%
  \pgfmathsinandcos\sinEl\cosEl{#2} % elevation
  \pgfmathsinandcos\sint\cost{#3} % latitude
  \pgfmathsetmacro\yshift{\cosEl*\sint}
  \tikzset{#1/.estyle={cm={\cost,0,0,\cost*\sinEl,(0,\yshift)}}} %
}
\newcommand\DrawLongitudeCircle[2][1]{
  \LongitudePlane{\angEl}{#2}
  \tikzset{current plane/.prefix style={scale=#1}}
  % angle of "visibility"
  \pgfmathsetmacro\angVis{atan(sin(#2)*cos(\angEl)/sin(\angEl))} %
  \draw[current plane] (\angVis:1) arc (\angVis:\angVis+180:1);
  \draw[current plane,dashed] (\angVis-180:1) arc (\angVis-180:\angVis:1);
}
\newcommand\DrawLatitudeCircleArrow[2][1]{
  \LatitudePlane{\angEl}{#2}
  \tikzset{current plane/.prefix style={scale=#1}}
  \pgfmathsetmacro\sinVis{sin(#2)/cos(#2)*sin(\angEl)/cos(\angEl)}
  % angle of "visibility"
  \pgfmathsetmacro\angVis{asin(min(1,max(\sinVis,-1)))}
  \draw[current plane,decoration={markings, mark=at position 0.6 with {\arrow{<}}},postaction={decorate},line width=.6mm] (\angVis:1) arc (\angVis:-\angVis-180:1);
  \draw[current plane,dashed,line width=.6mm] (180-\angVis:1) arc (180-\angVis:\angVis:1);
}
\newcommand\DrawLatitudeCircle[2][1]{
  \LatitudePlane{\angEl}{#2}
  \tikzset{current plane/.prefix style={scale=#1}}
  \pgfmathsetmacro\sinVis{sin(#2)/cos(#2)*sin(\angEl)/cos(\angEl)}
  % angle of "visibility"
  \pgfmathsetmacro\angVis{asin(min(1,max(\sinVis,-1)))}
  \draw[current plane] (\angVis:1) arc (\angVis:-\angVis-180:1);
  \draw[current plane,dashed] (180-\angVis:1) arc (180-\angVis:\angVis:1);
}
\newcommand\coil[1]{
  {\rh * cos(\t * pi r)}, {\apart * (2 * #1 + \t) + \rv * sin(\t * pi r)}
}
\makeatletter
\define@key{DrawFromCenter}{style}[{->}]{
  \tikzset{DrawFromCenterPlane/.style={#1}}
}
\define@key{DrawFromCenter}{r}[1]{
  \def\@R{#1}
}
\define@key{DrawFromCenter}{center}[(0, 0)]{
  \def\@Center{#1}
}
\define@key{DrawFromCenter}{theta}[0]{
  \def\@Theta{#1}
}
\define@key{DrawFromCenter}{phi}[0]{
  \def\@Phi{#1}
}
\presetkeys{DrawFromCenter}{style, r, center, theta, phi}{}
\newcommand*\DrawFromCenter[1][]{
  \setkeys{DrawFromCenter}{#1}{
    \pgfmathsinandcos\sint\cost{\@Theta}
    \pgfmathsinandcos\sinp\cosp{\@Phi}
    \pgfmathsinandcos\sinA\cosA{\angEl}
    \pgfmathsetmacro\DX{\@R*\cost*\cosp}
    \pgfmathsetmacro\DY{\@R*(\cost*\sinp*\sinA+\sint*\cosA)}
    \draw[DrawFromCenterPlane] \@Center -- ++(\DX, \DY);
  }
}
\newcommand*\DrawFromCenterText[2][]{
  \setkeys{DrawFromCenter}{#1}{
    \pgfmathsinandcos\sint\cost{\@Theta}
    \pgfmathsinandcos\sinp\cosp{\@Phi}
    \pgfmathsinandcos\sinA\cosA{\angEl}
    \pgfmathsetmacro\DX{\@R*\cost*\cosp}
    \pgfmathsetmacro\DY{\@R*(\cost*\sinp*\sinA+\sint*\cosA)}
    \draw[DrawFromCenterPlane] \@Center -- ++(\DX, \DY) node {#2};
  }
}
\makeatother

% not mandatory, but I though it was better to set it blank
\setbeamertemplate{headline}{}
\def\beamer@entrycode{\vspace{-\headheight}}

\tikzstyle{snakearrow} = [decorate, decoration={pre length=0.2cm,
  post length=0.2cm, snake, amplitude=.4mm,
  segment length=2mm},thick, ->]

%% document-wide tikz options and styles

\tikzset{%
  % >=latex, % option for nice arrows
  inner sep=0pt,%
  outer sep=2pt,%
  mark coordinate/.style={inner sep=0pt,outer sep=0pt,minimum size=3pt,
    fill=black,circle}%
}
\tikzset{
  % Define standard arrow tip
  >=stealth',
  % Define style for boxes
  punkt/.style={
    rectangle,
    rounded corners,
    draw=black, very thick,
    text width=8em,
    minimum height=2.5em,
    text centered},
}

\tikzset{onslide/.code args={<#1>#2}{%
    \only<#1>{\pgfkeysalso{#2}}
    % \pgfkeysalso doesn't change the path
  }}
\tikzset{alt/.code args={<#1>#2#3}{%
    \alt<#1>{\pgfkeysalso{#2}}{\pgfkeysalso{#3}}
    % \pgfkeysalso doesn't change the path
  }}
\tikzset{temporal/.code args={<#1>#2#3#4}{%
    \temporal<#1>{\pgfkeysalso{#2}}{\pgfkeysalso{#3}}{\pgfkeysalso{#4}}
    % \pgfkeysalso doesn't change the path
  }}

\makeatletter
\newbox\@backgroundblock
\newenvironment{backgroundblock}[2]{%
  \global\setbox\@backgroundblock=\vbox\bgroup%
  \unvbox\@backgroundblock%
  \vbox to0pt\bgroup\vskip#2\hbox to0pt\bgroup\hskip#1\relax%
}{\egroup\egroup\egroup}
\addtobeamertemplate{background}{\box\@backgroundblock}{}
\makeatother

\title{Building Single Molecules from Single Atoms}
\date{Sep 5, 2019}
\author[Yichao Yu]{Yichao Yu\\
  \vspace{0.5cm}
  {\footnotesize Lee Liu, Kenneth Wang, Lewis Picard, Jonathan Hood}\\
  {\footnotesize Jessie T. Zhang, Eliot Fenton, Yen-Wei Lin}}
\institute{Ni Group/Harvard}

\begin{document}

\pgfdeclarelayer{tweezer}
\pgfsetlayers{tweezer,main}
\pgfooclass{tweezer}{
  \method tweezer() {
  }
  \method drawTweezer(#1,#2,#3) {
    \begin{pgfonlayer}{tweezer}
      \shade[shading=radial,path fading=glow fading,shift={(#1,#2)},rotate=90,yscale=1,
      fill opacity=0.9,inner color=#3]
      plot[draw,samples=200,domain=-1.5:1.5] function {sqrt(0.01 + x**2 / 5)}
      -- plot[draw,samples=200,domain=1.5:-1.5] function {-sqrt(0.01 + x**2 / 5)};
    \end{pgfonlayer}
  }
  \method drawAtom(#1,#2,#3,#4) {
    \fill [#4,path fading=glow2 fading] (#1,#2) circle (#3);
  }
  \method drawNaAtom(#1,#2,#3) {
    \pgfoothis.drawAtom(#1,#2,#3,orange);
  }
  \method drawCsAtom(#1,#2,#3) {
    \pgfoothis.drawAtom(#1,#2,#3,blue);
  }
  \method drawNaTweezer(#1,#2) {
    \pgfoothis.drawTweezer(#1,#2,orange!35!black!30);
  }
  \method drawCsTweezer(#1,#2) {
    \pgfoothis.drawTweezer(#1,#2,blue!30!black!30);
  }
  \method up(#1,#2) {
    \pgfoothis.drawCsTweezer(#1,#2);
    \pgfoothis.drawNaAtom(#1,#2+0.06,0.12);
    \pgfoothis.drawCsAtom(#1,#2-0.06,0.16);
  }
  \method down(#1,#2) {
    \pgfoothis.drawCsTweezer(#1,#2);
    \pgfoothis.drawCsAtom(#1,#2+0.06,0.16);
    \pgfoothis.drawNaAtom(#1,#2-0.06,0.12);
  }
  \method naTrap(#1,#2) {
    \pgfoothis.drawNaTweezer(#1,#2);
    \pgfoothis.drawNaAtom(#1,#2,0.12);
  }
  \method csTrap(#1,#2) {
    \pgfoothis.drawCsTweezer(#1,#2);
    \pgfoothis.drawCsAtom(#1,#2,0.16);
  }
}
\pgfoonew \mytweezer=new tweezer()

{
  \usebackgroundtemplate{
    \makebox[\paperwidth][c]{\centering\includegraphics[width=\paperwidth]{front_bg.png}}
  }
  \setbeamercolor{title}{fg=white}
  \setbeamercolor{author}{fg=white}
  \setbeamercolor{institute}{fg=white}
  \setbeamercolor{date}{fg=white}
  \begin{frame}{}
    \titlepage
  \end{frame}
}

% I'm going to talk about our experiment on ....
% I was told that not everyone here is familar with molecules
% so I'll start with what we are doing and why we are using .....

% (1min)

% Most of you are probably familar with atoms
% * Laser cooling/trapping
% * Internal state control
% * High fidelity imagining
% * ... (and you can put in your favorate property of atoms here)
% There are way too many examples for this so here's just one result from
% Greiner group that demonstrates all the properties I mentioned.

% On the other hand, there's basically one problem with atoms
% * Too simple
% Weak interaction so need to try very hard to make a non-trivial system
% Few internal states so that the tricks that you can play with them is limited.

% (3min)

% Molecule
% * Interaction
% * Rich internal structure (more for polyatomic molecules)
% If we can have the same control

% Applications
% * Quantum information
% * Quantum simulations
% * Quantum chemistry
% * Precision measurement

% Spoiler: we cant achieve the same level of control yet
% That's what I'll talk about

% (5min)

\begin{frame}[t]{From Atom to Molecule}
  \begin{center}
    \begin{tikzpicture}
      \node at (-3, 0) {\LARGE Atom};
      \node[align=left, text width=6cm, below] at (-3, -0.5) {
        \begin{itemize}
        \item Laser cooling/trapping
        \item Internal state control
        \item High fidelity imagining
        \item ...
        \end{itemize}
      };
      \visible<2-3>{
        \node[below,align=center] at (-3, -3.5)
        {\includegraphics[width=4cm]{lithium_microscope}\\
          {\tiny Nature 545, 462-466 (2017)}};
      }
      \visible<3->{
        \node at (3, 0) {\LARGE Molecule};
        \node[align=left, text width=6cm, below] at (3, -0.5) {
          \begin{itemize}
          \item Strong interaction
          \item Rich internal structure
          \end{itemize}
        };
      }
      % TODO: applications
    \end{tikzpicture}
  \end{center}
\end{frame}

% As I said, cooling or in general controlling molecules is hard.
% There has been a few ways that people used to obtain ultracold molecules:
% * Direct cooling
% -- Hard because absense of cycling transition.
% -- But can be done on some molecules that sort of do.
% -- Pioneered by ... including here at Harvard in the Doyle group.
% * Making molecules from atoms
% -- Take advantage of atom cooling.
% -- Use a combination of magnetic and optical transitions.
% -- First realized by ....
% * Our approach
% -- Combine the bulk gas approach with recent progress on optical tweezers
% -- Do not rely as heavily on collisional properties for cooling and overlapping.
% -- Naturally single site resolution and manipulation
% -- Fast cycle time/high fidelity

% (8min)

\begin{frame}[t]{Path to Ultracold Molecules}
  \begin{center}
    \begin{tikzpicture}
      \visible<1-2> {
        \node at (-3, 0) {\textbf{\large Direct molecule cooling}};
        % TODO: picture/ref
      }
      \visible<2-> {
        \node at (3, 0) {\textbf{\large Making molecule from atoms}};
        % TODO: picture/ref
      }
      \visible<3-> {
        \node at (-3, 0) {\textbf{\large Assemble molecule in tweezers}};
        % TODO: picture/ref
      }
    \end{tikzpicture}
  \end{center}
\end{frame}

% So here's an outline for the rest of my talk.
% I'll first give an overview of out experiment, both the procedure and the apparatus
% Then I'll talk about a few intersting aspect of the atom part of our experiment
% i.e. some trapping and cooling issues that we've had
% which required us to develop new techniques for our setup.
% Finally, I'll talk about our progress on measuring the interaction between atoms
% and making molecules.
% (9min)
\begin{frame}{Outline}
  \tableofcontents
\end{frame}

\section{System Overview}

% As I said, in our experiment, we make molecules from atoms inside the optical tweezer
% so the obvious first step is to get the atoms in there.
% Load directly from the MOT and further cool in the tweezer which I'll talk about later.

% We load the Na and Cs atoms separately in two tweezers, that are ... apart.
% Obviously, serves no good to make molecules
% So we merge them by moving the beam.
% A few ways to do this but we use AOBD.

% After merge without killing or heating the atoms
% Try to make molecules or do some study on the interaction to help us make molecules

\begin{frame}{Steps}
  % TODO steps
  % TODO AOBD drawing
\end{frame}

% The whole experiment looks like this
% A Na MOT and for scale.
\begin{frame}{}
  % TODO pictures of the experiment
\end{frame}

% (12 min)

\section{Trapping and Cooling of Atoms}

% Our experiment starts with trapped atoms in tweezers.
% People have demonstrated before us.
% It sounds simple but turns out to be anything but.

% Well, Cs is fine.
% Got without too much effort.
% But Na is not.
% A few properties.
% * Low vapor pressure (fewer atoms to load from)
% * Broad linewidth (higher Doppler temperature)
% * Low mass (move faster and higher recoil velocity/energy)
% * Small HF structure (6 line width, worse PG cooling)

% Fine just need deeper trap but run into the real issue
% * Light shift
% In order to load, need cooling in the tweezer
% However, since we need such a deep trap, light shift turns off the cooling
% Got stuck in an awkward situation where a shallow trap can keep the cooling work but
% can't hold the atom while a deeper trap can hold the atom but the atom doesn't want to come in.

% Tried a bunch of ways (turn trap around atom, two wavelengths to cancel the trap)
% before Nick came up with a genius/crazy idea that people have used on large ODT.

% See laser cooling relies on scattering that happens at a rate of tens of MHz
% OTOH, the tweezer is only needed for trap motion that happens ~ few hundred kHz.
% Means that if we switch the tweezer on and off. .....

% Tried on Cs
% Essentially made a magic trap.

% Then tried on Na, it worked.
% Na live atom.

% Na and Cs 60 % loading on a good day.
% (20min)
\begin{frame}{Single Atom in Tweezer}
  % TODO ref of other tweezers
\end{frame}

% Since we want to control the state of the molecule by controlling the state of the atom
% including the motional state,
% we want to cool the atoms in the trap to the motional ground state in the tweezer.

% We use RSC

% Again, this has been done in other tweezers/optical lattice for some species.
% And you can probably guess which atom is harder.
\begin{frame}{Raman Sideband Cooling}
\end{frame}

\section{Atom-Atom Interaction and Molecule Formation}

\begin{frame}{Outline}
  \tableofcontents[currentsection]
\end{frame}

% So after getting both atom into a single quantum state,
% we can finally study their interaction and make molecules.

% The way we want to make molecule is by optical transfer.
% Essentially drive a transition from the atomic to the molecular state.
% This spare us of a huge FB coil although we are also working on that at the same time

% So we started in atomic state in the potential, here...
% Want to get to here since it's the state with the strongest dipole moment.
% The energy difference is huge and the wavefunction is dissimilar.
% So even though in an ideal world we might be able to do the transfer a single step,
% it's at least technically challenging to pull it off.

% Two steps
% First step (Raman) that replaces the FB association to transfer to a loosely bound state
% Then the same as previous experiement using a STIRAP.
% By splitting the two, we can do the slow inital transfer with a smaller energy difference
% and therefore easier to manage with the laser
% Do the big gap faster
\begin{frame}{Optical Transfer to Molecular State}
\end{frame}

% Raman transfer obviously need some information about the excited state
% So that's what we set out to measure first. (i.e. PA spectroscopy)

% We really have the most textbook system to do this measurement.
% We only have two atoms
% We know the initial state
% We shine some light on it, which might create some excited state molecule that
% falls down to some states that we can't see.
% We know the final state

% Full state info, improve signal-to-noise

% We start with near threshold since the lines are denser/easier to see.
% Single body - flat
% Two body - peaks
% Agreement with the theory/see new lines

% Also measured deeper lines after we found these lines
% and I might show it later if we have time.
\begin{frame}{Photoassociation (PA) Spectroscopy}
\end{frame}

% Now we know the excited state properties,
% we'd also like to know the ground state ones

% In AMO, an important quantity that governs the interaction in the ground state is.
% Useful for ....

\begin{frame}{Interaction Shift}
\end{frame}

%% Raman resonance
\begin{frame}{All-Optical Transfer to Weakly-Bound Molecular}
\end{frame}

\begin{frame}{}
\end{frame}

\begin{frame}{}
\end{frame}

\end{document}
